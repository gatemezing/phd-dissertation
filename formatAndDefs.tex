\usepackage{amsmath,amssymb}             % AMS Math
\usepackage[french]{babel}
%\usepackage[latin1]{inputenc}
%\usepackage[T1]{fontenc}
\usepackage[left=1.5in,right=1.3in,top=1.1in,bottom=1.1in,includefoot,includehead,headheight=13.6pt]{geometry}

%\usepackage[svgnames]{xcolor}


\renewcommand{\baselinestretch}{1.05}


% Table of contents for each chapter

\usepackage[nottoc, notlof, notlot]{tocbibind}
\usepackage{minitoc}
\setcounter{minitocdepth}{2}
\mtcindent=15pt
% Use \minitoc where tos put a table of contents

\usepackage{aecompl}

% Glossary / list of abbreviations

\usepackage[intoc]{nomencl}
\renewcommand{\nomname}{List of Abbreviations}

\makenomenclature

% My pdf code

\usepackage{ifpdf}

\ifpdf
  \usepackage[pdftex]{graphicx}
  \DeclareGraphicsExtensions{.jpg}
  \usepackage[a4paper,pagebackref,hyperindex=true]{hyperref}
\else
  \usepackage{graphicx}
  \DeclareGraphicsExtensions{.ps,.eps}
  \usepackage[a4paper,dvipdfm,pagebackref,hyperindex=true]{hyperref}
\fi

\graphicspath{{.}{images/}}

% Links in pdf
\usepackage{color}
\definecolor{linkcol}{rgb}{0,0,0.4} 
\definecolor{citecol}{rgb}{0.5,0,0} 

% Change this to change the informations included in the pdf file

% See hyperref documentation for information on those parameters

\hypersetup
{
bookmarksopen=true,
pdftitle="ThesisManuscript",
pdfauthor="Ghislain Auguste ATEMEZING", 
pdfsubject="Publication of geodata on the Web", %subject of the document
%pdftoolbar=false, % toolbar hidden
pdfmenubar=true, %menubar shown
pdfhighlight=/O, %effect of clicking on a link
colorlinks=true, %couleurs sur les liens hypertextes
pdfpagemode=None, %aucun mode de page
pdfpagelayout=SinglePage, %ouverture en simple page
pdffitwindow=true, %pages ouvertes entierement dans toute la fenetre
linkcolor=linkcol, %couleur des liens hypertextes internes
citecolor=citecol, %couleur des liens pour les citations
urlcolor=linkcol %couleur des liens pour les url
}

% definitions.
% -------------------

\setcounter{secnumdepth}{3}
\setcounter{tocdepth}{2}

% Some useful commands and shortcut for maths:  partial derivative and stuff

\usepackage [table]{xcolor}


\newcommand{\pd}[2]{\frac{\partial #1}{\partial #2}}
\def\abs{\operatorname{abs}}
\def\argmax{\operatornamewithlimits{arg\,max}}
\def\argmin{\operatornamewithlimits{arg\,min}}
\def\diag{\operatorname{Diag}}
\newcommand{\eqRef}[1]{(\ref{#1})}

\usepackage{rotating}                    % Sideways of figures & tables
%\usepackage{bibunits}
%\usepackage[sectionbib]{chapterbib}          % Cross-reference package (Natural BiB)
%\usepackage{natbib}                  % Put References at the end of each chapter
                                         % Do not put 'sectionbib' option here.
                                         % Sectionbib option in 'natbib' will do.
\usepackage{fancyhdr}                    % Fancy Header and Footer

% \usepackage{txfonts}                     % Public Times New Roman text & math font
  
%%% Fancy Header %%%%%%%%%%%%%%%%%%%%%%%%%%%%%%%%%%%%%%%%%%%%%%%%%%%%%%%%%%%%%%%%%%
% Fancy Header Style Options

\pagestyle{fancy}                       % Sets fancy header and footer
\fancyfoot{}                            % Delete current footer settings

%\renewcommand{\chaptermark}[1]{         % Lower Case Chapter marker style
%  \markboth{\chaptername\ \thechapter.\ #1}}{}} %

%\renewcommand{\sectionmark}[1]{         % Lower case Section marker style
%  \markright{\thesection.\ #1}}         %

\fancyhead[LE,RO]{\bfseries\thepage}    % Page number (boldface) in left on even
% pages and right on odd pages
\fancyhead[RE]{\bfseries\nouppercase{\leftmark}}      % Chapter in the right on even pages
\fancyhead[LO]{\bfseries\nouppercase{\rightmark}}     % Section in the left on odd pages

\let\headruleORIG\headrule
\renewcommand{\headrule}{\color{black} \headruleORIG}
\renewcommand{\headrulewidth}{1.0pt}
\usepackage{colortbl}
\arrayrulecolor{black}

\fancypagestyle{plain}{
  \fancyhead{}
  \fancyfoot{}
  \renewcommand{\headrulewidth}{0pt}
}

%\usepackage{algorithm}
%\usepackage{algorithmic}
%\usepackage[ruled,vlined]{algorithm2e}
%%% Clear Header %%%%%%%%%%%%%%%%%%%%%%%%%%%%%%%%%%%%%%%%%%%%%%%%%%%%%%%%%%%%%%%%%%
% Clear Header Style on the Last Empty Odd pages
\makeatletter

\def\cleardoublepage{\clearpage\if@twoside \ifodd\c@page\else%
  \hbox{}%
  \thispagestyle{empty}%              % Empty header styles
  \newpage%
  \if@twocolumn\hbox{}\newpage\fi\fi\fi}

\makeatother
 
%%%%%%%%%%%%%%%%%%%%%%%%%%%%%%%%%%%%%%%%%%%%%%%%%%%%%%%%%%%%%%%%%%%%%%%%%%%%%%% 
% Prints your review date and 'Draft Version' (From Josullvn, CS, CMU)
\newcommand{\reviewtimetoday}[2]{\special{!userdict begin
    /bop-hook{gsave 20 710 translate 45 rotate 0.8 setgray
      /Times-Roman findfont 12 scalefont setfont 0 0   moveto (#1) show
      0 -12 moveto (#2) show grestore}def end}}
% You can turn on or off this option.
% \reviewtimetoday{\today}{Draft Version}
%%%%%%%%%%%%%%%%%%%%%%%%%%%%%%%%%%%%%%%%%%%%%%%%%%%%%%%%%%%%%%%%%%%%%%%%%%%%%%% 

\newenvironment{maxime}[1]
{
\vspace*{0cm}
\hfill
\begin{minipage}{0.5\textwidth}%
%\rule[0.5ex]{\textwidth}{0.1mm}\\%
\hrulefill $\:$ {\bf #1}\\
%\vspace*{-0.25cm}
\it 
}%
{%

\hrulefill
\vspace*{0.5cm}%
\end{minipage}
}

\let\minitocORIG\minitoc
\renewcommand{\minitoc}{\minitocORIG \vspace{1.5em}}

\usepackage{multirow}
%\usepackage{slashbox}

\newenvironment{bulletList}%
{ \begin{list}%
	{$\bullet$}%
	{\setlength{\labelwidth}{25pt}%
	 \setlength{\leftmargin}{30pt}%
	 \setlength{\itemsep}{\parsep}}}%
{ \end{list} }

\newtheorem{definition}{D�finition}
\renewcommand{\epsilon}{\varepsilon}

% centered page environment

\newenvironment{vcenterpage}
{\newpage\vspace*{\fill}\thispagestyle{empty}\renewcommand{\headrulewidth}{0pt}}
{\vspace*{\fill}}
\usepackage{verbatim}
%\usepackage{minitoc}
%\usepackage{wrapfig}
\usepackage{listings}
\usepackage{hyperref}
\hypersetup{
    bookmarks=true,         % show bookmarks bar?
    unicode=false,          % non-Latin characters in Acrobat�s bookmarks
    pdftoolbar=true,        % show Acrobat�s toolbar?
    pdfmenubar=true,        % show Acrobat�s menu?
    pdffitwindow=false,     % window fit to page when opened
    pdfstartview={FitH},    % fits the width of the page to the window
    pdftitle={Publishing geodata on the web},    % title
    pdfauthor={Ghislain Auguste Atemezing},     % author
    pdfsubject={ThesisManuscript},   % subject of the document
    pdfcreator={Ghislain Auguste Atemezing},   % creator of the document
    pdfproducer={Atemezing}, % producer of the document
    pdfkeywords={geodata} {semantic web} {linked data}, % list of keywords
    pdfnewwindow=true,      % links in new window
    colorlinks=false,       % false: boxed links; true: colored links
    linkcolor=DarkBlue,          % color of internal links
    citecolor=green,        % color of links to bibliography
    filecolor=magenta,      % color of file links
    urlcolor=cyan           % color of external links
}

\newcommand{\todo}[1]{\textcolor{red}{%
\raisebox{-0.2em}[0pt][0pt]{\makebox[0pt]{\rule{1pt}{1.1em}}\makebox[0pt][l]{\raisebox{1em}{\rule{5ex}{1pt}}}}
\textbf{TODO:} \textit{#1}
\raisebox{-0.2em}[0pt][0pt]{\makebox[0pt][r]{\rule{5ex}{1pt}}\makebox[0pt]{\rule{1pt}{1em}}}%
}}

\newcommand{\comments}[1]{\textcolor{blue}{%
\raisebox{-0.2em}[0pt][0pt]{\makebox[0pt]{\rule{1pt}{1.1em}}\makebox[0pt][l]{\raisebox{1em}{\rule{5ex}{1pt}}}}
\textbf{COMMENTS:} \textit{#1}
\raisebox{-0.2em}[0pt][0pt]{\makebox[0pt][r]{\rule{5ex}{1pt}}\makebox[0pt]{\rule{1pt}{1em}}}%
}}

%\usepackage[dvips]{graphicx, color}
%\usepackage[bookmarks=true, bookmarksnumbered=true,hypertexnames=false, breaklinks=true]{hyperref}
%\DeclareGraphicsExtensions{.eps,.ps}

\newlength{\plarg}
\setlength{\plarg}{16cm}
\newlength{\glarg}
\setlength{\glarg}{17cm}
\lstset{
         basicstyle=\scriptsize\ttfamily, % Standardschrift
         numbers=left,               % Ort der Zeilennummern
         numberstyle=\tiny,          % Stil der Zeilennummern
         frame=single,
         %stepnumber=2,               % Abstand zwischen den Zeilennummern
         numbersep=5pt,              % Abstand der Nummern zum Text
         tabsize=2,                  % Groesse von Tabs
         extendedchars=true,         %
         breaklines=true,            % Zeilen werden Umgebrochen
         keywordstyle=\color{red},
                frame=b,         
  %       keywordstyle=[1]\textbf,    ]{\index{keywords, BEGIN}}% Stil der Keywords
  %       keywordstyle=[2]\textbf,    %
  %      keywordstyle=[3]\textbf,    %
 %        keywordstyle=[4]\textbf,   \sqrt{\sqrt{}} %
         stringstyle=\color{blue}\ttfamily, % Farbe der String
         showspaces=false,           % Leerzeichen anzeigen ?
         showtabs=false,             % Tabs anzeigen ?
         xleftmargin=17pt,
         framexleftmargin=17pt,
         framexrightmargin=5pt,
         framexbottommargin=4pt,
       	% backgroundcolor=\color{gray},
         showstringspaces=false,   
         language= C++, 
         captionpos=b, 
         caption=Platform Integrity Aspect, % Leerzeichen in Strings anzeigen ?        
 }


\listfiles
  
\setlength{\parindent}{0pt}

\newcommand\CyrGuillemot{%
  \def\selectguillfont{\fontencoding{OT2}\fontfamily{wncyr}\selectfont}
  \def\guillemotleft{\selectguillfont\symbol{60}}
  \def\guillemotright{\selectguillfont\symbol{62}}
}

\newcommand\PlGuillemot{%
  \def\selectguillfont{\fontencoding{OT4}\fontfamily{cmr}\selectfont}
  \def\guillemotleft{\selectguillfont\symbol{174}}
  \def\guillemotright{\selectguillfont\symbol{175}}
}

\newcommand\LaGuillemot{%
  \def\selectguillfont{\fontencoding{U}\fontfamily{lasy}%
    \fontseries{m}\fontshape{n}\selectfont}
  \def\guillemotleft{\selectguillfont\hbox{\symbol{40}%
    \kern-0.20em\symbol{40}}}
  \def\guillemotright{\selectguillfont\hbox{\symbol{41}%
    \kern-0.20em\symbol{41}}}
}

\newcommand\ECGuillemot{%
  \def\selectguillfont{\fontencoding{T1}\fontfamily{cmr}\selectfont}
  \def\guillemotleft{\selectguillfont\symbol{19}}
  \def\guillemotright{\selectguillfont\symbol{20}}
}

\newcommand\LMGuillemot{%
  \def\selectguillfont{\fontencoding{T1}\fontfamily{lmr}\selectfont}
  \def\guillemotleft{\selectguillfont\symbol{19}}
  \def\guillemotright{\selectguillfont\symbol{20}}
}

\newcommand\CyrGLeft{\CyrGuillemot\guillemotleft}
\newcommand\CyrGRight{\CyrGuillemot\guillemotright}
\newcommand\PlGLeft{\PlGuillemot\guillemotleft}
\newcommand\PlGRight{\PlGuillemot\guillemotright}
\newcommand\LaGLeft{\LaGuillemot\guillemotleft}
\newcommand\LaGRight{\LaGuillemot\guillemotright}
\newcommand\ECGLeft{\ECGuillemot\guillemotleft}
\newcommand\ECGRight{\ECGuillemot\guillemotright}
\newcommand\LMGLeft{\LMGuillemot\guillemotleft}
\newcommand\LMGRight{\LMGuillemot\guillemotright}



\newcommand{\algorithmicrequire}{\textbf{Require:}}
\newcommand{\algorithmicensure}{\textbf{Ensure:}}