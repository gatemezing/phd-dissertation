\chapter{Conclusions and Future Perspectives}
\label{ch:conc}
\begin{flushright}
\textit{``How do we know that Semantic Web technologies were actually better here, \\as opposed to being what the developers found most familiar?''} \\ (D. Karger)\footnote{\url{http://goo.gl/hQQ3h5}}
\end{flushright}


\section{Conclusions}
\label{sec:final}
%Write here summary of all the work in this document w.r.t the three parts.

This thesis is focused on the challenges of publishing geodata on the Web and a more generic approach to visualize data as Linked Data target to lay-users. The former considers the diversity of different formats used to publish legacy geospatial data, the different projections (or Coordinate Reference Systems) and the representation of complex geometries. The latter approach is different from the state-of-the-art in visualizations where the complexity of SPARQL and RDF is not sufficiently hidden from the users. A deep analysis of the literature has revealed some limitations in the publication of geospatial data and visualization tools, namely:
\begin{itemize}
\item Limited of complex geometries exposed in structured representation, instead of literals. 
\item The absence of an explicit reference to CRSs in direct georeference data on the Web.
\item Absence of visualization tool targeted to lay users to easily grasp the essence of the underlying data published as LOD.  
\item Many data silos for applications built and published on the Web, lost in many HTML pages.
\item Few tools that provide an integrated environment for publishing raw data into Linked Data, from data modeling until the final step of storing the dataset in an endpoint.
\item The difficulty for publishers to understand and check the compatibility of the licenses between vocabularies and datasets.
  
\end{itemize}

In this thesis, we have provided different vocabularies that all together support the publication of geodata integrating almost all the CRSs, extending the existing vocabularies. The vocabularies have been used to publish the French Administrative Units, with the data compatible with GeoSPARQL standards. Regarding the visualizations, after reviewing visual tools and existing applications on the Web, we have developed an ontology to better expose the data on the Web for better interoperability. Besides we have proposed a framework to generate automatically visualizations based on categories detected on datasets published as Linked Data, based on predefined categories used in InfoVis and mapped with vocabularies. 

\subsection{Review of the Contributions}
%\todo{get inspiration with Boris thesis}
This section reviews the main contributions of this thesis and the solutions to solved some of the open research problems in publishing and consuming data on the Semantic Web:

\begin{itemize}
\item The model and implementation of a vocabulary for geometry, topological entities and CRSs.
\item The implementation of an API for converting data between different CRSs accessible on the Web.
\item We have contributed in the development of the Datalift platform, an integrated environment to publish raw data on the Web.
\item The comparison of triple stores for geodata against the geometries handled (literal or structured) to assess which one to use when publishing geospatial data. 
\item The publication of the French Administrative Units available at \texttt{data.ign.fr} endpoint, based on the vocabularies developed and implemented. Moreover, we have provided interlinking with relevant existing geospatial datasets.
\item We have published the CRSs with unique URIs for better look up and integration in structured geometries on the Web.
\item The contribution to the \textit{French LOD (FrLOD)} cloud, with more datasets published using the Datalift platform, and covering the French territory.
\item We have proposed a generic approach to automatically generate visualizations based on predefined categories.
\item We have developed two innovative applications consuming events and statistical datasets
\item We have proposed a vocabulary and a tool that can improve the discovery of applications contests in Open Data events.
\item We have also proposed an approach to harmonize prefixes used in different catalogues of vocabulary.
\item We have developed new ranking metrics for vocabularies based on Information Content theories.
\item Finally, we have built a more efficient tool for checking licenses compatibility between vocabularies and datasets.
\end{itemize}


\section{Future Perspectives}
\label{sec:future}

In this thesis, we have tackled some open research problems within the context of publishing and consuming open data on the Web but there are still open issues and challenges for future work. We would like to mention some of the most important from our perspective, based on different aspects related to the workflow of publishing Linked Data, more specifically in the geospatial domain. 

\subsection{Opportunities and Challenges for IGN-France}
\label{sec:challenges}

 The need for interoperable reference geographic data to share and combine georeferenced environmental spatial information is particularly acknowledged by the INSPIRE Directive. The INSPIRE Directive \cite{inspire2009} aims to create a European Union (EU) spatial data infrastructure\footnote{\url{ http://inspire.jrc.ec.europa.eu/index.cfm/pageid/48}}. INSPIRE is based on a number of high level common principles, with some of them very closed to the key concepts of Semantic Web goals, and specifically the Linked Data principles. We provide below the correspondence of our contributions mapped to the five goals of INSPIRE:  
\begin{itemize}
\item \textbf{P1}: \textit{Data should be collected only once and kept where it can be maintained most effectively}. The use of good and stable URI policies can help achieve this principle. IGN as a French geospatial dataset provider, is committed to be provide accurate information, and so will be the URIs chosen for the experimental portal.
\item \textbf{P2}: \textit{It should be possible to combine seamless spatial information from different sources across Europe and share it with many users and applications}. This is more or less the goal of the interlinking tasks performed with other datasets on the wild. The models developed and well-documented can ease the conversion by other mapping agencies or institutions of their datasets.  
\item \textbf{P3}:  \textit{It should be possible for information collected at one level/scale to be shared with all levels/scales; detailed for thorough investigations, general for strategic purposes}. One of the drawback of the models proposed is that they don't currently admit many geometries attached to a feature. This will certainly be one of the extension foreseen for the models. However, the precise classifications for the features is a starter to fulfill this principle.
\item \textbf{P4}: \textit{Geographic information needed for good governance at all levels should be readily and transparently available}. Publishing \url{data.ign.fr} is one of the objective to have also data both in human and machine interpretable using semantic concepts.
\item \textbf{P5}: \textit{Easy to find what geographic information is available, how it can be used to meet a particular need, and under which conditions it can be acquired and used.} Publishing data on the web contribute \textit{per se} in leveraging their discovery and integration. Moreover, an explicit license attached to datasets published help achieving this principle.
\end{itemize}
For geographic data producers, the benefit of publishing their data on the Web according to Linked Data  (LD) principles is twofold. On the one hand, their data are interoperable with other published datasets and they can be referenced by external resources and used as spatial reference data, which would not have been straightforward when published according to spatial data infrastructures (SDI) standards. On the other hand, the use of semantic Web technologies can help addressing interoperability issues which are not solved yet by geographic information standards. 
Moreover, the French national mapping agency (IGN) has different types of license policies for accessing data from their professional portal\footnote{\url{http://professionnels.ign.fr/}}  (e.g., research purpose, commercial use, access on demand, etc.), with some of them not necessary ``open'' or free to access: (e.g., BD TOPO\circledR). Although there is a clear understanding of the benefits of publishing and interconnecting data on the web, ongoing investigations on how to combine licenses on datasets are under consideration at IGN. Two solutions are under investigation: 
%(i) different license policies attached to datasets and (ii) the use of a security access mechanism on top of the datasets granting access based on a predetermined configuration on named graphs and resources. 
\begin{itemize}
\item Different license policies attached to given datasets: Here the attached license is given directly when published. So for example, if it is an open license, the endpoint is publicly available to be queried without any restriction.
\item The use of a security access mechanism on top of the datasets granting access according to a predetermined configuration list of named graphs, resources and operations allowed. This solution goes along with the work of Rotolo et al.\cite{rotolo2013deontic}, where even if there is an endpoint, a module for configuring the types of queries to perform and access policies have to be defined for subsets with special care to take into account compositions of licenses in the results.
\end{itemize}
According to Linked Data principles URIs should remain stable, even if administrative units change or disappear. This implies adapting the data vocabulary in order to handle data versioning and real world evolutions. This issue will be addressed in a future work, as we plan to release a spatio-temporal dataset describing the evolution of communes since the French Revolution. Another issue deals with the automation of the whole publication process, from traditional geographic data to fully interconnected RDF data.
The last issue deals with the use of multiple geometries for describing a geographic feature: geometries with different levels of detail, different CRS, different representation choices. This has been superficially addressed in our use case with the use of both polygons and points for representing respectively the surface and the centroid of departments, but should be further investigated for both query answering and map design purposes.


\subsection{Generic Visualizations on Linked Data}
We plan to use a more exhaustive set of vocabularies in our generic queries for detecting those categories, plugging directly the wizard to the LOV catalogue. Regarding the aggregation properties, it can be extended to take into account other semantic relations (e.g: \texttt{skos:exactMatch}). Additionally, we plan to make an evaluation of the prototype and compare it to related tools such as the ones aiming to build a dataset profile. We also need to quantify when a category is ``important'' within a dataset. For example, is it enough for a dataset to be classified GEODATA with ten triples containing location ? From which number of triples found could be assign the categories and hence the visualizations? These issues can further be investigated to find the best trade-off.  Another drawback of our work on visualizations is the lack of users' evaluation, with sound experiments to understand users' needs, more focusing on the semantic aspects than the Web. A natural follow-up is to go through this evaluations and re-adapt the applications/visualizations based on the results. 
 
\subsection{Vocabularies and LOV}
\label{sec:nextSteps}
Regarding the harmonization of prefixes, the work can be extended in several directions. Sticking to the two services we have studied and already contributed to harmonize, the possible next steps would be to automate as far as possible the tasks that have been made semi-automatically so far: \emph{i)} developing a unique interface for submitting namespaces and prefixes to both services; \emph{ii)} bridging the LOV back-office and the prefix-cc database using both services API in order to publish a list of common recommended prefixes. The latter goes beyond the limited framework of the two original services since such a list could be consolidated and endorsed by the main actors in vocabulary publication and management, and recommended for use in linked data applications. This could be picked up by the upcoming W3C Vocabulary Management Working Group as part of the new Data Activity\footnote{\url{http://www.w3.org/2013/05/odbp-charter.html}}.

%\todo{here is the perspective for ranking voabs} \\
 As per ranking vocabularies, we aim to take into account the equivalence axioms (between classes and properties) when computing the Information Content, and more generally, all sort of semantic relationships between terms. Also, we plan to compare our ranking model with other ranking approaches such as graph-based ones (e.g., pagerank). Another future direction work is to investigate the dependency ranking between vocabularies, by focusing on a specific type of ``inlinks'' (i.e. extensions, generalization) and study how they affect the PIC values.


We have made the assumption in this thesis that the access to data was either by  querying the SPARQL endpoint, or by browsing or by downloading the dumps. Recently, it is emerging a new way of accessing the data on the Web: through the triple pattern fragments\footnote{\url{http://linkeddatafragments.org/}}. Linked Data Fragments \cite{verborgh2014ldf} aims at exploring interfaces to 
solve queries at the client side with server data. 
Servers can offer data at low processing cost in a way that enables client-side querying. Thus, moving intelligence from the server to the client. One possible direction of study could be to use this concept for evaluating endpoints consuming only structured geometries versus literal for real live applications. Finally, triple fragments concepts can be apply to detect also patterns for visualization in different endpoints.
 
% \todo{discuss here if it could be wise to use Linked Data Fragments... maybe as future work?}

 