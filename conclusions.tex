\chapter{Conclusions and Future Perspectives}
\label{ch:conc}

\section{Conclusions}
\label{sec:final}
Write here summary of all the work in this document w.r.t the three parts.

\section{Future Perspectives}
\label{sec:nextSteps}
Regarding the harmonization of prefixes, the work can be extended in several directions. Sticking to the two services we have studied and already contributed to harmonize, the possible next steps would be to automate as far as possible the tasks that have been made semi-automatically so far: \emph{i)} developing a unique interface for submitting namespaces and prefixes to both services; \emph{ii)} bridging the LOV back-office and the prefix-cc database using both services API in order to publish a list of common recommended prefixes. The latter goes beyond the limited framework of the two original services since such a list could be consolidated and endorsed by the main actors in vocabulary publication and management, and recommended for use in linked data applications. This could be picked up by the upcoming W3C Vocabulary Management Working Group as part of the new Data Activity\footnote{\url{http://www.w3.org/2013/05/odbp-charter.html}}.

\todo{here is the perspective for ranking voabs} \\
 As future work, we aim to take into account the equivalence axioms (between classes and properties) when computing the Information Content, and more generally, all sort of semantic relationships between terms. Also, we plan to compare our ranking model with other ranking approaches such as graph-based ones (e.g. pagerank). Another future direction work is to investigate the dependency ranking between vocabularies, by focusing on a specific type of ``inlinks'' (i.e. extensions, generalization) and study how they affect the PIC values.
 
 \todo{plan regarding LD Wizard}
 We plan to use a more exhaustive set of vocabularies in our generic queries for detecting those categories, plugging directly the wizard to the LOV catalogue. Regarding the aggregation properties, it can be extended to take into account other semantic relations (e.g: \texttt{skos:exactMatch}). Additionally, we plan to make an evaluation of the prototype and compare it to related tools such as the ones aiming to build a dataset profile.