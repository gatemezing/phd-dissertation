\chapter{Glossary}

\section{List of Acronyms and Abbreviations}
\begin{description}

\myitem{AD:} {Average Delivery Delay}
\myitem{AP:} {Access Point}
\myitem{CCN:} {Content Centric Networking}
\myitem{CDF:} {Cumulative Distribution Function}
\myitem{DHCP:} {Dynamic Host Configuration Protocol}
\myitem{DNS:} {Domain Name System}
\myitem{DONA:} {Data Oriented Network Architecture}
\myitem{DTN:} {Delay or Disruption Tolerant Networks}
\myitem{DYMO:} {Dynamic MANET On-demand Routing Protocol}
\myitem{EID:} {Endpoint Identifier}
\myitem{ER:} {Encounter-based Replication}
\myitem{ESAR:} {Encounter and Social Affiliation-based Replication}
\myitem{ESS:} {Extended Service Set}
\myitem{GUID:} {Globally Unique Identifier}
\myitem{GW:} {Gateway}
\myitem{HA:} {Home Agent}
\myitem{HeNNA:} {Heterogeneous Networks Naming Architecture}
\myitem{HIP:} {Host Identity Protocol}
\myitem{HNA:} {Host and Network Association}
\myitem{LISP:} {Locator/Identifier Separation Protocol}
\myitem{MANET:} {Mobile Ad-hoc Networks}
\myitem{MDR:} {Message Delivery Ratio}
\myitem{MeDeHa:} {Message Delivery in Heterogeneous Disruption-prone Networks}
\myitem{MN:} {Mobile Node}
\myitem{NAT:} {Network Address Translation}
\myitem{OLSR:} {Optimized Link State Routing Protocol}
\myitem{P2P:} {Peer-to-Peer}
\myitem{PDA:} {Personal Digital Assistant}
\myitem{PSM:} {Power Saving Mode}
\myitem{RWP:} {Random Waypoint Mobility Model}
\myitem{SAR:} {Social Affiliation-based Replication}
\myitem{SID:} {Session Identifier}
\myitem{VANET:} {Vehicular Ad-hoc Networks}

\end{description}

\section{Basic Definitions}

\begin{description}
\myitem{Association:} {Connection of a node with an infrastructure-based network.}
\myitem{Delay/Disruption Tolerant Networks:} {Networks that tolerate nodes intermittent connectivity and are not based on end-to-end Internet principle.}
\myitem{Deterministic Routing:} {The encounters between two nodes can be determined based on their route information.}
\myitem{Disassociation:} {Disconnection of a node from an infrastructure-based network.}
\myitem{Enforced Routing:} {Special-purpose nodes are added to the network to enhanced routing.}
\myitem{Forwarding:} {Only one copy of a message exists in the network.}
\myitem{Gateway (GW):} {A node that runs the MeDeHa software and has the capability to connect to multiple networks simultaneously.}
\myitem{Handoff:} {Connection transfer of a mobile node from one AP to another within an ESS.}
\myitem{Hello Handshake:} {Neighbor sensing mechanism of MeDeHa for ad-hoc networks.}
\myitem{Hop-by-hop Reliability:} {The data transfer between two neighboring nodes is reliable.}
\myitem{Infrastructure-based Networks:} {Networks with fixed infrastructure and connectivity to the backbone. Examples include Wifi, WiMax, cellular-based networks.}
\myitem{Infrastructure-based Node:} {A basestation or an AP providing the backbone connectivity to wireless nodes.}
\myitem{Infrastructure-less Networks:} {Ad-hoc Networks without any fixed infrastructure including multi-hop mobile ad-hoc networks or MANETs.}
\myitem{Late Binding:} {The process of acquiring the routing address of a destination from its application-level identifier while the packet is being routed.}
\myitem{Opportunistic Routing:} {The encounters between two nodes are not known a priori.}
\myitem{Replication:} {Multiple copies per message exist in the network.}
\myitem{Ubiquitous Networks:} {Networks that provide continuous connectivity everywhere.}

\end{description}

\noindent \HRule\\[-2.35ex]
\noindent \Hslim\\[0ex]