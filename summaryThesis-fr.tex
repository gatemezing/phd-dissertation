\chapter{R\'{e}sum\'{e} en Francais} \label{chap:resume}

Au cours de ces derni\`{e}res ann\'{e}es, le domaine de l'Open Data a reçu une attention croissante de la part des administrations publiques qui veulent tirer avantage de la publication de donn\'{e}es ouvertes sur le Web. Les b\'{e}n\'{e}fices suppos\'{e}s de cette ouverture pour les citoyens font r\'{e}f\'{e}rence à une meilleure transparence dans les prises de d\'{e}cisions publiques, à une meilleure gouvernance ou encore au d\'{e}veloppement d'un \'{e}co-syst\`{e}me num\'{e}rique qui tirerait un profit \'{e}conomique des applications analysant ces donn\'{e}es. Cependant, la r\'{e}alit\'{e} montre que la simple ouverture et la publication de donn\'{e}es par les administrations ne sont pas suffisantes au regard des d\'{e}fis li\'{e}s à la vari\'{e}t\'{e} des formats (XML, CSV, Excel, PDF, Shape), des m\'{e}thodes d'acc\`{e}s (API, base de donn\'{e}es) et à l'absence de nomenclature qui permettrait une meilleure r\'{e}utilisation et interconnexion avec d'autres jeux de donn\'{e}es. Dans cette th\`{e}se, nous explorons comment l'utilisation des standards et des technologies du web s\'{e}mantique peut aider à r\'{e}soudre les probl\`{e}mes caus\'{e}s par l'h\'{e}t\'{e}rog\'{e}n\'{e}it\'{e} et la diversit\'{e} des formats de donn\'{e}es et des structures de repr\'{e}sentations dans le domaine g\'{e}ographique.

Cette th\`{e}se applique les principes des ``donn\'{e}es li\'{e}es'' dans le domaine de l'information g\'{e}ographique, un domaine clef pour les administrations publiques qui couvrent, par d\'{e}finition, un territoire. En particulier, nous traitons de trois aspects essentiels dans le workflow de traitement et de publication de donn\'{e}es g\'{e}o-spatiales et de leur consommation (visualisation), avec des sc\'{e}narios d'utilisation issus de l’Institut Nationale de l’Information G\'{e}ographique et Foresti\`{e}re (IGN) : (1) Comment repr\'{e}senter efficacement et stocker des donn\'{e}es g\'{e}ospatiales sur le Web pour assurer des applications interop\'{e}rables ? (2) Quelles sont les meilleures options pour un utilisateur pour interagir avec des donn\'{e}es s\'{e}mantiques interconnect\'{e}es ? (3) Quels m\'{e}canismes peuvent \^{e}tre mis en place pour aider à la pr\'{e}servation des donn\'{e}es structur\'{e}es de haute qualit\'{e} sur le Web?

Nos contributions sont structur\'{e}es en trois grandes parties correspondantes aux probl\'{e}matiques susmentionn\'{e}es, avec des applications sp\'{e}cifiques dans le domaine g\'{e}ographique. Nous proposons et d\'{e}veloppons trois vocabulaires pour repr\'{e}senter des syst\`{e}mes de coordonn\'{e}es de r\'{e}f\'{e}rence (CRS), des entit\'{e}s topographiques et la g\'{e}om\'{e}trie associ\'{e}e à ces entit\'{e}s. Ces ontologies \'{e}tendent d'autres vocabulaires existants et ajoutent deux avantages suppl\'{e}mentaires : l’utilisation explicite de CRS identifi\'{e}s par des URIs pour repr\'{e}senter la g\'{e}om\'{e}trie, et la capacit\'{e} de d\'{e}crire des g\'{e}om\'{e}tries structur\'{e}es en RDF. Nous avons ainsi publi\'{e} la base de donn\'{e}es GEOFLA, en utilisant et en contribuant à la plate-forme Datalift, un outil permettant de convertir et publier des donn\'{e}es brutes en donn\'{e}es li\'{e}es. Nous avons \'{e}galement \'{e}valu\'{e} de mani\`{e}re syst\'{e}matique la performance des points d'acc\`{e}s SPARQL pour traiter des requ\^{e}tes spatiales.

Concernant la ``consommation'' de donn\'{e}es RDF, apr\`{e}s avoir examin\'{e} les diff\'{e}rentes cat\'{e}gories des outils de visualisation (g\'{e}n\'{e}riques et sp\'{e}cifiques à des jeux de donn\'{e}es), nous proposons un vocabulaire pour d\'{e}crire les applications de visualisation (DVIA). En outre, nous formalisons et mettons en œuvre un workflow pour visualiser des donn\'{e}es s\'{e}mantiques interconnect\'{e}es à travers l'outil LDVizWiz, un assistant de visualisation g\'{e}n\'{e}rique de donn\'{e}es li\'{e}es sur le Web.

La derni\`{e}re partie de la th\`{e}se d\'{e}crit des contributions au catalogue des vocabulaires li\'{e}es (LOV) ainsi qu'une proposition originale pour utiliser LOV avec une m\'{e}thodologie de cr\'{e}ation d'ontologie telle que NeOn dans le but d'am\'{e}liorer la r\'{e}utilisation des vocabulaires. Nous proposons une heuristique pour aligner les vocabulaires et un classement de ces derniers en fonction de m\'{e}triques li\'{e}es au contenu de l'information des termes d\'{e}finis dans les vocabulaires. Enfin, la th\`{e}se apporte des r\'{e}ponses sur la façon de v\'{e}rifier la compatibilit\'{e} des licences entre les vocabulaires et les jeux de donn\'{e}es dans le workflow de publication. Tout au long de la th\`{e}se, nous d\'{e}montrons les avantages de l'utilisation des technologies s\'{e}mantiques et des standards du W3C pour mieux d\'{e}couvrir, interconnecter et visualiser les donn\'{e}es g\'{e}ospatiales gouvernementales pour leur publication sur le Web.

\section{Questions de recherche}
Dans cette th\`{e}se, nous proposons des solutions pour les d\'{e}fis li\'{e}s à la publication des donn\'{e}es g\'{e}ographique sur le Web des donn\'{e}es, qui sont les suivantes :
\begin{enumerate}
\item \textit{Vocabulaires :} Comment mod\'{e}liser l'information g\'{e}ographique sur le Web ? Comment \'{e}valuer les ontologies du domaine g\'{e}ographique ? Comment s\'{e}rialiser les g\'{e}om\'{e}tries complexes dans un environnement comme le Web ?
\item \textit{Languages de requ\^{e}tes :} Comment \'{e}crire des requ\^{e}tes efficaces qui ciblent les donn\'{e}es g\'{e}ospatiales sur le Web ? Comment stocker et indexer les g\'{e}odonn\'{e}es repr\'{e}sent\'{e}es en RDF ?
\item \textit{Donn\'{e}es :} Comment extraire et convertir les g\'{e}odonn\'{e}es pour publication sur le Web ? Quelles sont les bonnes pratiques pour repr\'{e}senter des g\'{e}om\'{e}tries complexes sur le Web ? Comment int\'{e}grer pleinement la compatibilit\'{e} des syst\`{e}mes de coordonn\'{e}es sur des jeux de donn\'{e}es ?
\item \textit{Publication :} Comment d\'{e}velopper des environnements qui passent à l'\'{e}chelle pour couvrir le chaîne de publication des g\'{e}odonn\'{e}es ? Quels sont les triples stores appropri\'{e}es pour le stockage des g\'{e}odonn\'{e}es ? Quelles sont les m\'{e}triques à utiliser pour l'interconnexion de diff\'{e}rentes ressources de g\'{e}odonn\'{e}es sur le Web ?
\item \textit{Applications et interfaces utilisateurs :} Comment g\'{e}n\'{e}rer des visualisations de donn\'{e}es g\'{e}ospatiales li\'{e}es entre elles ? Quels sont les API de haut niveau appropri\'{e}es qui facilitent le d\'{e}veloppement d'interfaces utilisateur pour les donn\'{e}es g\'{e}ospatiales ? Pouvons-nous r\'{e}utiliser les outils de cartographie existants tels que Google Maps, Bing Maps ou OpenSteetMap ?
\end{enumerate}

Dans cette th\`{e}se, nous abordons les enjeux de publication des donn\'{e}es tant du point de vue des \'{e}diteurs que de celui des utilisateurs. Les \'{e}diteurs et les utilisateurs ont besoin de solutions pragmatiques qui les aident à choisir un vocabulaire, trouver un outil pour convertir des fichiers Shape selon des vocabulaires existants, puis transformer en RDF et publier les donn\'{e}es suivant des bonnes pratiques.

Apr\`{e}s la publication du jeux de donn\'{e}es sur le Web, les \'{e}diteurs et les utilisateurs doivent comprendre ces donn\'{e}es pendant que les d\'{e}veloppeurs doivent pouvoir cr\'{e}er des applications. Le Web commence à contenir de plus en plus de donn\'{e}es structur\'{e}es, qui ne sont pas toujours exploit\'{e}es par les utilisateurs finaux, à cause de la complexit\'{e} du mod\`{e}le RDF et de son langage de requ\^{e}te, SPARQL. Ainsi, il est important de cr\'{e}er des visualisations pour explorer, analyser et montrer les b\'{e}n\'{e}fices du Linked Data aux non-experts. Dans ce processus, il existe des questions de recherches à r\'{e}soudre telles que :
\begin{itemize}
 \item Comment trouver des visualisations adapt\'{e}es selon les jeux de donn\'{e}es tout en masquant la complexit\'{e} du language de requ\^{e}tes SPARQL ?
 \item Quelles sont les propri\'{e}t\'{e}s importantes pour visualiser les ressources du Web, en fonction du domaine et des attentes des utilisateurs ?
 \item Comment combler le foss\'{e} entre les outils traditionnels existants de visualisation de l'information, la plupart du temps aux formats CSV/XLS, JSON ou formats propri\'{e}taires pour int\'{e}grer facilement le mod\`{e}le de donn\'{e}es RDF en entr\'{e}e ?
 \item Comment d\'{e}velopper des applications interop\'{e}rables sur les catalogues de donn\'{e}es gouvernementaux en Open Data ? Comment r\'{e}utiliser les applications existantes ?
\end{itemize}

Tout en essayant de r\'{e}pondre aux d\'{e}fis ci-dessus mentionn\'{e}s, nous exposons l'\'{e}tat de l'art et les approches existantes dans le domaine des visualisations de donn\'{e}es li\'{e}es.

\section{Contributions}
\label{sec:contribs}

Les contributions de cette th\`{e}se sont organis\'{e}e en trois parties principales : la mod\'{e}lisation et la publication des donn\'{e}es g\'{e}ospatiales, la visualisation de donn\'{e}es et des applications sur le Web et la contribution dans les standards.

\subsection{Mod\'{e}lisation et publication des donn\'{e}es g\'{e}ospatiales}
La g\'{e}olocalisation est cruciale pour de nombreuses applications tant pour agents humains que les logiciels. De plus en plus des masses de donn\'{e}es sont ouvertes et interconnect\'{e}es sur le Web. Une mod\'{e}lisation des donn\'{e}es g\'{e}ographiques de mani\`{e}re efficace en r\'{e}utilisant autant que possible des ontologies ou des vocabulaires existants qui d\'{e}crivent à la fois les fonctionnalit\'{e}s g\'{e}ospatiales et leurs formes. Dans la premi\`{e}re partie de notre travail, nous examinons diff\'{e}rentes approches de mod\'{e}lisation utilis\'{e}es dans les syst\`{e}mes d'information g\'{e}ographique (SIG) et la communaut\'{e} des donn\'{e}es ouvertes (LOD). Notre objectif est de contribuer aux efforts r\'{e}els dans la repr\'{e}sentation des objets g\'{e}ographiques avec des attributs tels que l'emplacement, les points d'int\'{e}r\^{e}t (POI), et les adresses sur le Web de donn\'{e}es. Nous nous concentrons sur le territoire français et nous fournissons des exemples de vocabulaires repr\'{e}sentatifs qui peuvent \^{e}tre utilis\'{e}s pour d\'{e}crire les objets g\'{e}ographiques. Nous proposons quelques alignements entre diff\'{e}rents vocabulaires (DBpedia, schema.org, LinkedGeoData, Foursquare, etc.) afin de permettre l'interop\'{e}rabilit\'{e} tout en interconnectant les g\'{e}odonn\'{e}es en France avec d'autres jeux de donn\'{e}es.

Concernant cet aspect de notre recherche, nos contributions sont les suivantes :
\begin{enumerate}
  \item Nous avons propos\'{e} et d\'{e}velopp\'{e} une ontologie d\'{e}crivant les caract\'{e}ristiques et les points d'int\'{e}r\^{e}t pour le territoire français, en r\'{e}utilisant une taxonomie existante (GeOnto) en l'alignant sur d'autres vocabulaires connexes dans le domaine de la g\'{e}olocalisation.
  \item  Nous avons \'{e}tudi\'{e} comment \'{e}tendre les vocabulaires existants dans le domaine g\'{e}ographique afin de prendre en compte une mod\'{e}lisation efficace des g\'{e}om\'{e}tries complexes. Ce faisant, nous abordons les questions de repr\'{e}sentation de g\'{e}om\'{e}trie complexe dans le Web de donn\'{e}es, d\'{e}crivant l'\'{e}tat de mise en oeuvre des fonctions g\'{e}ospatiales dans triples stores et une comparaison avec la nouvelle norme GeoSPARQL. Nous faisons enfin quelques recommandations et plaidons pour la r\'{e}utilisation des vocabulaires plus structur\'{e}es pour la publication d'entit\'{e}s topographiques pour mieux r\'{e}pondre aux exigences des donn\'{e}es issues de IGN-France.
  \item Nous avons fait une \'{e}tude comparative des triples stores, comparant leur capacit\'{e} de stockage des informations spatiales et leur impl\'{e}mentation des fonctions topologiques ra rapport à celles d\'{e}jà existantes dans les normes de l'Open Geospatial Consortium (OGC)\footnote{\url{http://www.opengeospatial.org/}}.
  \item  Nous avons conçu et d\'{e}velopp\'{e} des vocabulaires pour d\'{e}crire les g\'{e}om\'{e}tries complexes avec diff\'{e}rents syst\`{e}mes de coordonn\'{e}es, avec application directe aux unit\'{e}s administratives françaises.
  \item Nous avons interconnect\'{e} des g\'{e}odonn\'{e}es du contexte français avec des jeux de donn\'{e}es g\'{e}ospatiales existantes sur le Web, tels que LinkedGeodata, GADM, NUTS et Geonames.
  \item Nous avons contribu\'{e} à la cr\'{e}ation du ``nuage donn\'{e}es'' ( LOD Cloud) repr\'{e}sentant la publication de 8 jeux de donn\'{e}es, soit 340 millions de triplets couvrant le territoire français.
\end{enumerate}

Consommer des donn\'{e}es sur le Web grâce à des visualisations comporte autant de d\'{e}fis que les applications doivent de conformer à la structure du graphe RDF, la s\'{e}mantique sous-jacente du jeu de donn\'{e}es et de l'interaction homme-machine pour comprendre facilement de quoi traitent ces donn\'{e}es.

\subsection{Outils de visualisation des donn\'{e}es gouvernementales li\'{e}es}
\label{visu}
Nous \'{e}tudions d'abord quelques applications innovantes qui ont \'{e}t\'{e} d\'{e}velopp\'{e}es sur des jeux de donn\'{e}es publi\'{e}es en Open Data par les gouvernements (Royaume-Uni, USA, France) et des administrations locales. Nous avons ensuite d\'{e}riv\'{e} et propos\'{e} 8 cas d'utilisation (sc\'{e}narios) qui peuvent \^{e}tre d\'{e}velopp\'{e}es pour consommer des donn\'{e}es provenant des diff\'{e}rents fournisseurs principaux en France: INSEE, DILA, IGN, FING, etc. Nous mentionnons que les cas d'utilisation les plus int\'{e}ressants sont ceux qui montrent la valeur ajout\'{e}e des jeux de donn\'{e}es interconnect\'{e}es. Ces sc\'{e}narios d\'{e}velopp\'{e}s et d\'{e}ploy\'{e}s, peuvent \^{e}tre utiles pour montrer les avantages de donn\'{e}es li\'{e}es dans une vari\'{e}t\'{e} de domaines tels que l'\'{e}ducation, le tourisme, le patrimoine culturel, les administrations civiles, les tribunaux, la m\'{e}decine, etc.

En ce qui concerne les outils utilis\'{e}s pour la visualisation, nous avons identifi\'{e} et classer en deux cat\'{e}gories, en fournissant pour chacun d'eux des exemples pertinents: (i) - des outils qui fonctionnent sur des donn\'{e}es RDF, et (ii) des outils qui fonctionnent sur d'autres formats structur\'{e}s. Nous proposons donc des crit\`{e}res de base pour \'{e}valuer un outil de visualisation de donn\'{e}e en g\'{e}n\'{e}ral, avec des poids attach\'{e}s à chaque crit\`{e}re.

Nos contributions sur la visualisation sont les suivantes :
\begin{enumerate}
 \item Nous avons construit une application des \'{e}lections pr\'{e}sidentielles du premier tour français en 2012 en utilisant les donn\'{e}es de \url{http://data.gouv.fr} et d'autres institutions publiques. L'application disponible à \url{http://www.eurecom.fr/~atemezin/DemoElection/} a \'{e}t\'{e} construit avec l'outil Exhibit. Il vise à mettre en valeur l'int\'{e}gration des jeux de donn\'{e}es h\'{e}t\'{e}rog\`{e}nes: les r\'{e}sultats politiques en CSV, le taux de chômage, les donn\'{e}es des candidats, les informations des d\'{e}partements de France provenant des donn\'{e}es DBpedia. L'utilisateur peut filtrer par image du candidat, le taux de chômage et par d\'{e}partement pour voir les scores, avec des informations plus enrichies sur le d\'{e}partement.
 \item Nous avons mis en place un outil g\'{e}n\'{e}rique pour explorer les g\'{e}odonn\'{e}es sur une carte, en fonction de la d\'{e}tection automatique des donn\'{e}es via de requ\^{e}tes SPARQL dans le nuage LOD contenant des jeux de donn\'{e}es g\'{e}ospatiales.
 \item Nous avons d\'{e}velopp\'{e} une application consommatrice de donn\'{e}es g\'{e}ospatiales et statistiques combinant plusieurs jeux de donn\'{e}es dans l'\'{e}ducation de provenant du portail \url{http://data.gouv.fr} .
 \item Nous avons d\'{e}velopp\'{e} une application sur les \'{e}v\'{e}nements dans une conf\'{e}rence avec leurs m\'{e}dias supports r\'{e}concili\'{e}s provenant de nombreuses plates-formes sociales (Instagram, Twitter, etc.).
 \item Nous avons impl\'{e}ment\'{e} un vocabulaire pour structurer les applications sur le Web de donn\'{e}es. Le vocabulaire peut \^{e}tre utilis\'{e} pour d\'{e}couvrir des outils visuels ou graphiques utilis\'{e}s pour cr\'{e}er des applications.
 \item Nous avons mpl\'{e}ment\'{e} un plugin g\'{e}n\'{e}rique pour annoter des applications d\'{e}ve- lopp\'{e}es pour des hack-athon pouvant \^{e}tre inclus dans une page web, permettant la g\'{e}n\'{e}ration de contenu structur\'{e} de pages Web en utilisant le vocabulaire d\'{e}velopp\'{e}.
 \item Nous avons mis en place un assistant qui analyse un jeu de donn\'{e}es RDF et  recommande une visualisation bas\'{e}e sur des cat\'{e}gories pr\'{e}d\'{e}finies, en utilisant des requ\^{e}tes SPARQL g\'{e}n\'{e}riques pour faciliter l'exploration des jeux de donn\'{e}es publi\'{e}s sur le LOD.
\end{enumerate}

\subsection{Contributions aux standards}
\label{sec:contrib-standard}
Nous avons contribu\'{e} aux activit\'{e}s du groupe du W3C sur les donn\'{e}es gouvernementales li\'{e}es de travail (GLD WG)\footnote{\url{http://www.w3.org/2011/gld/}} de juillet 2011 jusqu'à d\'{e}cembre 2013. L'objectif du Groupe de travail \'{e}tait de ``fournir des normes et d'autres informations qui aident les gouvernements à travers le monde dans la publication de leurs donn\'{e}es aussi efficace qu'utilisable à l'aide des technologies du Web s\'{e}mantique''.

Nous avons contribu\'{e} à trois groupes de travail, et en particulier à deux documents :
\begin{enumerate}
 \item Un glossaire\footnote{\url{http://www.w3.org/TR/ld-glossary/}} pour la description des termes utilis\'{e}s dans le domaine du Linked data pour les potentiels producteurs et consommateurs de donn\'{e}es gouvernementales sur le Web
 \item Un document sur les bonnes pratiques de publication des donn\'{e}es gouvernementales sur le Web\footnote{\url{http://www.w3.org/TR/ld-bp/}}
\end{enumerate}

En ce qui concerne l'utilisation de vocabulaires standards, nous avons contribu\'{e} à :
\begin{itemize}
 \item Proposer une m\'{e}thode pour harmoniser les pr\'{e}fixes sur le Web de donn\'{e}es avec deux services: Linked Open vocabulaires (LOV)\footnote{\url{http://lov.okfn.org/dataset/lov/}} prefix.cc\footnote{\url{http://prefix.cc}}. le premier service est actuellement un catalogue à jour des vocabulaires utilis\'{e}s sur le Web, tandis que le dernier est un service pour les d\'{e}veloppeurs pour choisir, valider et chercher des pr\'{e}fixes pour leurs ressources ou ontologies. L'approche propos\'{e}e peut \^{e}tre \'{e}tendue à tout le catalogue du vocabulaire tant que les vocabulaires remplissent les conditions pour \^{e}tre ins\'{e}r\'{e}es dans le catalogue LOV.
 \item Concevoir et mettre en œuvre une nouvelle m\'{e}thode de classement des vocabulaires sur la base des m\'{e}triques du contenu de l'information et de l'information partitionn\'{e}e.
 \item Nous avons d\'{e}velopp\'{e} un outil qui d\'{e}termine en temps r\'{e}el si les diff\'{e}rentes licences pr\'{e}sentes dans un jeu de donn\'{e}es et les vocabulaires associ\'{e}s sont soit compatible ou non.
\end{itemize}

\section*{Plan de la Th\`{e}se}
\label{sec:thesis-structure}
Dans la premi\`{e}re partie de cette th\`{e}se, nous nous concentrons sur l'\'{e}tude des diff\'{e}rents mod\`{e}les et vocabulaires pour repr\'{e}senter la g\'{e}ographie et de la g\'{e}om\'{e}trie. Nous \'{e}tudions les points d'acc\`{e}s aux donn\'{e}es et d\'{e}crivons les probl\`{e}mes particuliers tels que les syst\`{e}mes de coordonn\'{e}es, et mettons en \'{e}vidence nos contributions dans ce dommaine: cr\'{e}ation de nouveaux vocabulaires r\'{e}utilisant les vocabulaires existants, impl\'{e}mentation d'un convertisseur en ligne entre des diff\'{e}rents syst\`{e}mes de coordon\'{e}es, etc. Nous d\'{e}crivons \'{e}galement comment des jeux de donn\'{e}es g\'{e}ographiques peuvent ensuite \^{e}tre convertis en RDF en utilisant le processu d'\'{e}l\'{e}vation des donn\'{e}es du projet Datalift afin de leur publication sur le Web. Nous montrons ensuite comment ces jeux de donn\'{e}es peuvent \^{e}tre align\'{e}es entre elles et concluons par une analyse approfondie de ces alignements dans le cas des jeux de donn\'{e}es de cartographie française fournis par l'Institut G\'{e}ographique et Foresti\`{e}re (IGN -France ).

\textbf{Le Chapitre 1} d\'{e}crit les limites actuelles de repr\'{e}sentation des g\'{e}odonn\'{e}es sur le Web et notre contribution sur les diff\'{e}rents vocabulaires pour repr\'{e}senter les g\'{e}om\'{e}tries, les syst\`{e}mes de coordonn\'{e}es de r\'{e}f\'{e}rence et les ressources topographiques. Nous proposons \'{e}galement des bonnes pratiques pour la publication des donn\'{e}es g\'{e}ospatiales sur le Web.

\textbf{Le Chapitre 2} met l'accent sur les outils de publication et des requ\^{e}tes d'interrogation des g\'{e}odonn\'{e}es, leurs diff\'{e}rences et leurs applications. Nous d\'{e}crivons la plate-forme Datalift, une plate-forme ouverte servant de catalyseur des sources de donn\'{e}es brutes vers des donn\'{e}es s\'{e}mantiques et interconnect\'{e}s. Apr\`{e}s avoir compar\'{e} Datalift avec Geoknow, nous l'appliquons dans le processus de publication d'unit\'{e}s administratives et le Gazetteer français. Nous pr\'{e}sentons ensuite l'\'{e}tat du nuage français LOD (FrLOD) des donn\'{e}es li\'{e}es et des exemples de requ\^{e}tes sur des g\'{e}om\'{e}tries structur\'{e}es publi\'{e}es dans le point d'acc\`{e}s \url{http://data.ign.fr}.

Dans la deuxi\`{e}me partie de la th\`{e}se, nous couvrons trois principales questions relatives à la façon de pr\'{e}senter les donn\'{e}es en RDF aux utilisateurs finaux. Tout d'abord, nous pr\'{e}sentons l'\'{e}tat de l'art des outils et des solutions pour la repr\'{e}sentation visuelle et l'exploration des donn\'{e}es en RDF (Visualbox, LODSpeaKr, Map4RDF, le mod\`{e}le ``Linked Data Visualization'', etc.). Ensuite, nous pr\'{e}sentons notre contribution: un assistant pour faciliter les visualisations automatique des points d'acc\`{e}s aux donn\'{e}es sur le Web, y compris le vocabulaire sp\'{e}cifiant les visualisations et le prototype impl\'{e}ment\'{e}. Par la suite, nous pr\'{e}sentons deux applications dans les domaines \'{e}v\'{e}nementiel et statistique pour mettre en exergue de mani\`{e}re innovante la r\'{e}utilisation des jeux de donn\'{e}es li\'{e}s. Enfin, nous impl\'{e}mentons un algorithme permettant de r\'{e}v\'{e}ler les propri\'{e}t\'{e}s les plus ``importantes'' des entit\'{e}s des ressources pour leur visualisation en partant de la Base de Connaissance de Google (GKP) ainsi qu'une \'{e}valuation faite sur les pr\'{e}f\'{e}rences des utilisateurs.

\textbf{Le Chapitre 3} fournit une revue de litt\'{e}rature sur des outils de visualisation et les applications, avec leurs limites. Nous d\'{e}crivons \'{e}galement l'\'{e}tat de l'art des applications sur le Web et proposons une classification des ``Applications de donn\'{e}es li\'{e}es''.

\textbf{Le Chapitre 4} pr\'{e}sente notre contribution sur de nouvelles approches pour g\'{e}n\'{e}rer des visualisations et des applications. Nous proposons tout d'abord une nouvelle approche pour les visualisations bas\'{e}es sur des cat\'{e}gories. Nous montrons ensuite une application dans le domaine g\'{e}ographique. Deux applications li\'{e}es aux \'{e}v\'{e}nements et aux statistiques sont \'{e}galement d\'{e}crits. Enfin, nous proposons la façon d'am\'{e}liorer la d\'{e}couverte d'applications dans les \'{e}v\'{e}nements Open Data grâce à un mod\`{e}le et un plugin universel pour annoter des pages Web en RDFa.

Dans la derni\`{e}re partie de la th\`{e}se dans le \textbf{Chapitre 5}, nous d\'{e}crivons diverses contributions aux vocabulaires ouverts li\'{e}s (description du catalogue, les publications des vocabulaires, des API et des points d'acc\`{e}s) : l'harmonisation des pr\'{e}fixes des vocabulaire, les m\'{e}triques pour classer les vocabulaire  en utilisant le contenu de l'information.

Dans le \textbf{Chapitre 6}, nous pr\'{e}sentons quelques id\'{e}es sur la v\'{e}rification de compatibilit\'{e} des licence entre les vocabulaires et les jeux de donn\'{e}es en utilisant la logique d\'{e}ontique en cr\'{e}ant un outil en ligne pour la d\'{e}tection automatique des licences sur es donn\'{e}es du Web.

Dans le \textbf{Chapitre 7}, nous concluons en mettant en \'{e}vidence certaines limites et perspectives pour de nouvelles directions de recherche.

\section*{Partie I: Int\'{e}gration des donn\'{e}es geo-spatiales sur le Web}

\subsection*{Chapitre I}
Dans ce chapitre, nous faisons une revue de la litterature des formats et des diff\'{e}rents vocabulaires utilis\'{e}s pour mod\'{e}liser des donn\'{e}es g\'{e}ospatiales sur le Web, en distinguant deux types de g\'{e}or\'{e}f\'{e}rencement: direct et indirect. Ensuite, nous identifions certaines limitations li\'{e}s à l'absence d'une r\'{e}f\'{e}rence explicite du Systeme de coordonn\'{e} g\'{e}ographique (SCG) dans les jeux de donn\'{e}es actuellement publi\'{e}s sur le Web. Nous proposons ensuite un service REST pour la conversion entre diff\'{e}rents SCG pour aider les \'{e}diteurs à \^{e}tre capable de g\'{e}rer diff\'{e}rentes projections dans les jeux donn\'{e}es. En outre, nous proposons et impl\'{e}mentons trois vocabulaires pour les g\'{e}om\'{e}tries, les SCG et les entit\'{e}s topographiques qui sont en ligne aux adresses respectives à \url{http://data.ign.fr/def/geometrie}, \url{http://data.ign.fr/def/ignf} et \url{http://data.ign.fr/def/topo}. Les vocabulaires \'{e}tendent ceux existants et int\`{e}grent deux avantages suppl\'{e}mentaires: un usage explicite de SCG identifi\'{e}s par des URIs pour la g\'{e}om\'{e}trie et la capacit\'{e} à d\'{e}crire des g\'{e}om\'{e}tries structur\'{e}es en RDF. Certains de nos r\'{e}sultats et la description du mod\`{e}le sont en cours de discussion pour la standardisation au W3C, comme par exemple \'{e}tendre le standard  GeoSPARQL pour int\'{e}grer de mani\`{e}re plus explicite les coordonn\'{e}es g\'{e}ographiques.

\subsection*{Chapitre II}
Dans ce chapitre, nous pr\'{e}sentons une \'{e}tude des outils d'extraction et de conversion de donn\'{e}es g\'{e}ospatiales en RDF. Ensuite, nous d\'{e}crivons \texttt{GeomRDF}, un outil d\'{e}velopp\'{e} au sein du projet Datalift qui va au-delà de l'\'{e}tat de l'art en fournissant des g\'{e}om\'{e}tries structur\'{e}es et conformes au standard GeoSPARQL. En outre, nous pr\'{e}sentons les limites des mod\`{e}les de donn\'{e}es existants en sugg\'{e}rant des recommandations aux \'{e}diteurs de g\'{e}odonn\'{e}es sur les aspects de stockage de gros volumes de donn\'{e}s. De m\^{e}me, une description d\'{e}taill\'{e}e de l'outil Datalift utilis\'{e} pour publier des donn\'{e}es sur le Web est fournie, avec une attention particuli\`{e}re sur notre contribution à la construction du nuage des donn\'{e}es du Linked Open Data sur des donn\'{e}es du territoire français avec des jeux de donn\'{e}es en 4-5 \'{e}toiles selon les principes de donn\'{e}es li\'{e}es. Enfin, nous montrons quelques cas d'utilisation du monde r\'{e}el des requ\^{e}tes SPARQL faisant usage tour à tour de la g\'{e}om\'{e}trie structur\'{e}e ou des fonctions g\'{e}ospatiales int\'{e}gr\'{e}es dans le triple store. Selon les besoins des utilisateurs et les jeux de donn\'{e}es sous-jacentes, l'utilisateur peut choisir entre la simplicit\'{e} du languqge de requ\^{e}te SPARQL, avec des limitations au niveau du triple store (par exemple, lors de l'usage des fonctions g\'{e}ospatiales int\'{e}gr\'{e}es), ou l'expressivit\'{e} du du vocabulaire que nous proposons (\texttt{geom}), comme crit\`{e}re dans le choix du triple store et du stockage des donn\'{e}es g\'{e}ospatiales.

\section*{Partie II: Visualisation des graphes de donn\'{e}es sur le Web}

\subsection*{Chapitre III}
Dans ce chapitre, nous d\'{e}crivons les diff\'{e}rents outils utilis\'{e}s pour la visualisation des donn\'{e}es structur\'{e}es et des graphes. Nous discutons \'{e}galement de diff\'{e}rents types d'applications actuellement construites bas\'{e}es sur des initiatives de donn\'{e}es ouvertes du gouvernement en Open Data. Le but de cet \'{e}tat de l'art est de proposer de nouvelles approches de g\'{e}n\'{e}ration et des outils de visualisations et des applications sur le Web de donn\'{e}es. Nous avons conçu et mis en œuvre un vocabulaire, \texttt{DVIA}, qui vise à mod\'{e}liser des applications pour plus d'interop\'{e}rabilit\'{e} et de d\'{e}couverte d'applications et d'outil de visualisations  sur le Web.

\subsection*{Chapitre IV}
Dans ce chapitre, nous avons pr\'{e}sent\'{e} une approche pour cr\'{e}er des visualisations au-dessus de donn\'{e}es li\'{e}es bas\'{e}s sur les technologies du Web s\'{e}mantique. Nous avons d'abord d\'{e}fini sept cat\'{e}gories des entit\'{e}s qui peuvent \^{e}tre associ\'{e}s à la visualisation des jeux de donn\'{e}es, et nous proposons de les mapper à d'autres vocabulaires de domaine. Nous pr\'{e}sentons ensuite une description des principales composantes d'un assistant (wizard) de visualisation dans le contexte de Linked Data. Nous d\'{e}crivons une impl\'{e}mentation en JavaScript comme \textit{preuve-de-concept} de notre proposition, avec les avantages d'\^{e}tre disponible en ligne et extensible. Nous pensons qu'un tel outil peut \^{e}tre facilement int\'{e}gr\'{e} dans une chaîne globale de  publication donn\'{e}es sur le Web, tels que Datalift ou GeoKnow. En outre, nous avons effectu\'{e} des exp\'{e}riences sur le graphe de connaissances de Google pour d\'{e}tecter des propri\'{e}t\'{e}s importantes à visualiser dans des entit\'{e}s, et avons \'{e}valu\'{e} en fonction aux pr\'{e}f\'{e}rences des utilisateurs. Ensuite, nous avons pr\'{e}sent\'{e} deux applications dans le domaine des statistiques et des \'{e}v\'{e}nements, consommant diff\'{e}rents jeux de donn\'{e}es en RDF sur le sc\'{e}nario des donn\'{e}s du monde r\'{e}el. Nous avons discut\'{e} de la façon d'am\'{e}liorer les applications d\'{e}velopp\'{e}es dans des contextes de hackathon, en proposant un vocabulaire et un outil pour peupler le mod\`{e}le en utilisant un plugin universel. Des exemples d'\'{e}v\'{e}nements pass\'{e}s ont d\'{e}jà \'{e}t\'{e} transform\'{e}s de mani\`{e}re semi-automatique utilisant à la fois le vocabulaire et le plugin.

\section*{Partie III: Contribution au catalogue des vocabulaires li\'{e}s}

\subsection*{Chapitre V}
Nous avons pr\'{e}sent\'{e} dans ce chapitre notre contribution au catalogue des vocabulaires ouverts et li\'{e}s, comme partie d'impl\'{e}mentations des avantages de l'utilisation de LOV dans la cr\'{e}ation et la gestion de l'ontologie (cas de la m\'{e}thodologie de NeOn), de l'harmonisation des pr\'{e}fixes et l'alignement des vocabulaires publi\'{e}s sur le Web, ou sur du ranking des vocabulaires en utilisant la th\'{e}orie du contenu de l'information. En appliquant ce dernier aux vocabulaires, nous avons essay\'{e} d'utiliser les fonctionnalit\'{e}s que nous jugeons ``pertinents''  à prendre en compte lorsque l'on veut des vocabulaires (par exemple: les jeux de donn\'{e}es r\'{e}utilis\'{e}s, les liens vers les vocabulaires externes). Nous comparons notre approche avec d'autres classements qui sont principalement bas\'{e}es sur la `` popularit\'{e} '' des vocabulaires. Ce travail peut ouvrir la voie vers une \'{e}valuation des vocabulaires avec des applications dans une approche plus syst\'{e}mique de recommandation des classes ou propri\'{e}t\'{e}s dans la gestion de l'ontologie, ou dans des applications de visualisation afin de proposer la  propri\'{e}t\'{e} la plus appropri\'{e}e à visualiser dans les ressource RDF contenant un grand nombre de propri\'{e}t\'{e}s.

\subsection*{Chapitre VI}
Dans ce chapitre, nous avons pr\'{e}sent\'{e} un outil en ligne pour v\'{e}rifier la compatibilit\'{e} entre des jeux de donn\'{e}es et des vocabulaires bas\'{e}s sur la logique ``RDF-defeasible'' de SPINdle. Nous avons impl\'{e}ment\'{e} le framework LIVE pour tester la compatibilit\'{e} des licences sur les donn\'{e}es publi\'{e}s sur le Web. Le but de ce framework est de v\'{e}rifier la compatibilit\'{e} des licences associ\'{e}es aux vocabulaires utilis\'{e}es pour g\'{e}n\'{e}rer un jeu de donn\'{e}es RDF et la licence associ\'{e}e au jeu de donn\'{e}es final. Plusieurs aspects d'ordre plus juridique doivent \^{e}tre pris en compte pour les travaux futurs. Plus pr\'{e}cis\'{e}ment, nous consid\'{e}rons que les vocabulaires comme des donn\'{e}es à part enti\`{e}re, mais ce n'est pas la seule interpr\'{e}tation possible. Par exemple, nous pouvons voir des vocabulaires comme une sorte de compilateur, de telle sorte qu'apr\`{e}s la cr\'{e}ation du jeu de donn\'{e}es, les vocabulaires externes ne sont plus utilis\'{e}es. Dans ce cas, quel serait le moyen appropri\'{e} pour d\'{e}finir un syst\`{e}me de v\'{e}rification de compatibilit\'{e} ? Comme travail futur, nous \'{e}tudierons en profondeur cette question ainsi que nous ferons une \'{e}valuation sur la facilit\'{e} d'utilisation de l'outil en ligne LIVE pour am\'{e}liorer l'interface utilisateur.

\section*{Conclusion et Perspectives}
\label{sec:conc}
Cette th\`{e}se est consacr\'{e}e aux d\'{e}fis de la publication des donn\'{e}es g\'{e}ospatiales sur le Web et une approche plus g\'{e}n\'{e}rique de visualiser les donn\'{e}es li\'{e}es pour les utilisateurs. La premi\`{e}re consid\`{e}re la diversit\'{e} des diff\'{e}rents formats utilis\'{e}s pour publier les donn\'{e}es g\'{e}ospatiales propri\'{e}taires, les diff\'{e}rentes projections (ou syst\`{e}mes de coordonn\'{e}es de r\'{e}f\'{e}rence) et la repr\'{e}sentation des g\'{e}om\'{e}tries complexes. Cette derni\`{e}re approche est diff\'{e}rente de l'\'{e}tat-de-l'art dans les visualisations où la complexit\'{e} du langage SPARQL et RDF est pas suffisamment cach\'{e}e des utilisateurs. Une analyse approfondie de la litt\'{e}rature a r\'{e}v\'{e}l\'{e} certaines limites dans la publication des donn\'{e}es g\'{e}ospatiales et des outils de visualisation, à savoir :
\begin{itemize}
 \item Une pr\'{e}sence limit\'{e}e des g\'{e}om\'{e}tries complexes repr\'{e}sent\'{e}es de mani\`{e}re structur\'{e}e, au lieu de litt\'{e}raux.
 \item L'absence d'une r\'{e}f\'{e}rence explicite aux SCG dans les donn\'{e}es au g\'{e}or\'{e}f\'{e}rencement direct sur le Web.
 \item L'absence d'outil de visualisation destin\'{e} aux utilisateurs permettant de comprendre facilement l'essence des donn\'{e}es sous-jacentes publi\'{e}es en LOD.
 \item Beaucoup de silos de donn\'{e}es pour les applications publi\'{e}es sur le Web, perdues dans de nombreuses pages HTML.
 \item Peu d'outils qui fournissent un environnement int\'{e}gr\'{e} pour la publication des donn\'{e}es brutes en donn\'{e}es li\'{e}es, partant la mod\'{e}lisation de donn\'{e}es jusqu'à l'\'{e}tape finale de stockage du jeu de donn\'{e}es dans un store RDF.
 \item La difficult\'{e} pour les \'{e}diteurs de donn\'{e}es de comprendre et de v\'{e}rifier la compatibilit\'{e} des licences entre les vocabulaires et les jeux de donn\'{e}es qu'ils r\'{e}utilisent venant du LOD.
\end{itemize}

Dans cette th\`{e}se, nous avons fourni des vocabulaires qui aident à la mod\'{e}lisation et la publication des donn\'{e}es g\'{e}ospatiales int\'{e}grant la quasi-totalit\'{e} des SCG, qui \'{e}tendent les vocabulaires existants. Les vocabulaires ont \'{e}t\'{e} utilis\'{e}s pour publier les unit\'{e}s administratives françaises, avec les donn\'{e}es compatibles au standard GeoSPARQL. En ce qui concerne les visualisations, apr\`{e}s avoir examin\'{e} des outils visuels et les applications existantes sur le Web, nous avons d\'{e}velopp\'{e} une ontologie pour mieux exposer les donn\'{e}es sur le Web pour une meilleure interop\'{e}rabilit\'{e}. Nous avons \'{e}galement propos\'{e} un framework pour g\'{e}n\'{e}rer automatiquement des visualisations bas\'{e}es sur les cat\'{e}gories d\'{e}tect\'{e}s sur des jeux de donn\'{e}es li\'{e}es et publi\'{e}es, en utilisant les cat\'{e}gories pr\'{e}d\'{e}finies de haut niveau utilis\'{e}es dans la taxonomie de la visualisation de l'information.

\subsection*{Revue des contributions}
Cette section examine les principales contributions de cette th\`{e}se et les solutions que nous avons apport\'{e} comme contributions dans le contexte de la publication des donn\'{e}es g\'{e}ospatiales sur le Web. Nos contributions d\'{e}crites tout au long de cette th\`{e}se sont les suivantes :

\begin{itemize}
 \item Nous avons mod\'{e}lis\'{e} et impl\'{e}ment\'{e} un vocabulaire pour la g\'{e}om\'{e}trie, les entit\'{e}s topologiques et les syst\`{e}mes de coordonn\'{e}es g\'{e}ographiques.
 \item Nous avons mis en place une API pour convertir des donn\'{e}es en ligne entre les diff\'{e}rents syst\`{e}mes de coordonn\'{e}es accessibles sur le Web.
 \item Nous avons publi\'{e} les diff\'{e}rents syst\`{e}mes de projections utilis\'{e}s en France avec des URI uniques pour am\'{e}liorer la recherche et l'int\'{e}gration des g\'{e}om\'{e}tries structur\'{e}es sur le Web.
 \item Nous avons contribu\'{e} à l'\'{e}laboration de la plate-forme Datalift, un environnement int\'{e}gr\'{e} de publication des donn\'{e}es brutes de formats h\'{e}t\'{e}rog\`{e}nes sur le Web.
 \item Nous avons fourni une comparaison des triples stores pour les g\'{e}odonn\'{e}es en projectant les types de g\'{e}om\'{e}tries nativement incorpor\'{e}es (litt\'{e}ral ou structur\'{e}e) pour aider à la recommandation lors de la publication des donn\'{e}es g\'{e}ospatiales.
 \item Nous avons publi\'{e} les donn\'{e}es sur les circonscriptions administratives françaises selon les bonnes pratiques du LOD accessible au \url{http://data.ign.fr} bas\'{e}es sur les vocabulaires que nous avons d\'{e}velopp\'{e}. En outre, nous avons fourni des alignements avec des jeux de donn\'{e}es g\'{e}ospatiales pertinentes existantes, tel que Geonames, GADM, NUTS, INSEE, etc.
 \item Nous avons publi\'{e} en RDF 15 millions d'adresses provenant d'Open Street Map France en utilisant le vocabulaire des adresses propos\'{e} par le W3C.
 \item Nous avons contribu\'{e} à la cr\'{e}ation du nuage de donn\'{e}es dans le contexte  français (FrLOD), en utilisant la plateforme Datalift, ainsi que des alignements avec des jeux de donn\'{e}es existantes. Ces donn\'{e}es ont la principale caract\'{e}ristique de couvrir la France.
 \item Nous avons revu la litt\'{e}rature et avons classifi\'{e} les applications construites sur des portails des donn\'{e}es ouvertes des gouvernements, et avons propos\'{e} un vocabulaire pour annoter s\'{e}mantiquement et am\'{e}liorer la recherche et l'extraction d'applications cr\'{e}es dans le cadres des hackathon sur les donn\'{e}es en Open Data.
 \item Nous avons propos\'{e} une approche g\'{e}n\'{e}rique pour g\'{e}n\'{e}rer automatiquement des visualisations bas\'{e}es sur des cat\'{e}gories pr\'{e}d\'{e}finies à l'aide des requ\^{e}tes SPARQL.
 \item Nous avons impl\'{e}ment\'{e} et \'{e}valu\'{e} une approche pour d\'{e}terminer les propri\'{e}t\'{e}s qui conviennent le mieux à utiliser pour choisir une entit\'{e} à visualiser, bas\'{e} sur une approche similaire à celle mise en oeuvre dans panel de recherche de la base de connaissance Google.
 \item Nous avons d\'{e}velopp\'{e} deux applications innovantes consommant des des donn\'{e}es \'{e}v\'{e}nementiels et statistiques mutualis\'{e}es avec des donn\'{e}es externes pr\'{e}sentes dans le nuage des donn\'{e}es li\'{e}es.
 \item Nous avons propos\'{e} un plugin g\'{e}n\'{e}rique pour am\'{e}liorer la d\'{e}couverte des applications construites dans les hackathons sur les donn\'{e}es en Open Data
 \item Nous avons \'{e}galement propos\'{e} une approche pour harmoniser les pr\'{e}fixes utilis\'{e}s dans les diff\'{e}rents catalogues de vocabulaire, avec une \'{e}valuation faite dans le cas des vocabulaires  sur Linked Open Vocabulary (LOV).
 \item Nous avons d\'{e}velopp\'{e} de nouvelles mesures de ranking pour les vocabulaires bas\'{e}es sur le contenu des informations et appliqu\'{e} dans LOV.
 \item Enfin, nous avons construit un outil plus efficace pour v\'{e}rifier la compatibilit\'{e} des licences entre les vocabulaires et les jeux de donn\'{e}es.
\end{itemize}

\subsection*{Perspectives}
\label{sec:future}
Dans cette th\`{e}se, nous avons abord\'{e} certains probl\`{e}mes ouverts de recherche dans le cadre de la publication et la consommation (visualisation) des donn\'{e}es ouvertes sur le Web, mais il reste encore des questions en suspens et des d\'{e}fis pour des travaux futurs. Nous mentionnons quelques-uns des plus importants dans la section suivante, bas\'{e}s sur diff\'{e}rents aspects li\'{e}s à la chaîne d'\'{e}dition des donn\'{e}es li\'{e}es, plus sp\'{e}cifiquement dans le domaine g\'{e}ospatial.

\subsubsection*{Opportunit\'{e}s et d\'{e}fis pour IGN-France}
\label{sec:challenges}
Le besoin de donn\'{e}es de r\'{e}f\'{e}rence g\'{e}ographiques interop\'{e}rables pour partager et combiner des informations environnementales spatiales g\'{e}or\'{e}f\'{e}renc\'{e}es est mis en \'{e}vidence par la directive europ\'{e}enne INSPIRE. La directive INSPIRE vise à cr\'{e}er dans l'espace de l'Union Europ\'{e}enne (UE) une infrastructure de donn\'{e}es spatiales\footnote{\url{http://inspire.jrc.ec.europa.eu/index.cfm/pageid/48}}. INSPIRE est bas\'{e}e sur un certain nombre de principes communs de haut niveau, avec certains d'entre eux tr\`{e}s propres des principes cl\'{e}s appliqu\'{e}s dans les fondements du Web s\'{e}mantique, et en particulier dans son impl\'{e}mentations dans les donn\'{e}es ouvertes et li\'{e}es. Nous fournissons ci-dessous la correspondance de nos contributions ayant un lien avec les cinq objectifs de la directive INSPIRE :\footnote{La traduction des objectifs est faite par nos propres soins.}
\begin{itemize}
 \item \textbf{P1}: \textit{Les donn\'{e}es doivent \^{e}tre collect\'{e}es une seule fois et conserv\'{e}es où elles peuvent \^{e}tre maintenues le plus efficacement possible}. L'utilisation de bonnes politiques et des URIs stables peuvent aider à atteindre ce principe. IGN comme un fournisseur de donn\'{e}es g\'{e}ospatiales en France est commis à des informations exactes, ainsi que seront donc les URIs choisis et utilis\'{e}s pour le portail s\'{e}mantique.
 \item \textbf{P2}: \textit{Il devrait \^{e}tre possible de combiner des informations spatiales transparente provenant de diff\'{e}rentes sources à travers l'Europe et les partager avec de nombreux utilisateurs et applications}. Ce principe est plus ou moins l'objectif des tâches d'interconnexion avec d'autres jeux de donn\'{e}es dans le Web. Il faudrait restreindre les domaines de recherche dans les donn\'{e}es europ\'{e}enes. Les mod\`{e}les d\'{e}velopp\'{e}s et bien document\'{e}s peuvent faciliter la conversion des donn\'{e}es par d'autres organisations ou institutions utilisant ou produisant les donn\'{e}es cartographiques.
 \item \textbf{P3}:  \textit{Il devrait \^{e}tre possible pour l'information recueillie à un niveau d'\'{e}chelle \^{e}tre partag\'{e}e à tous les autres niveaux ou \'{e}chelles; d\'{e}taill\'{e}e pour des recherches approfondies et g\'{e}n\'{e}rale à des fins strat\'{e}giques}. Un des inconv\'{e}nients des mod\`{e}les propos\'{e}s est qu'ils n'admettent pas actuellement de nombreuses g\'{e}om\'{e}tries attach\'{e}es à une entit\'{e} g\'{e}ographique ou adresse. Ce sera certainement l'une des extensions pr\'{e}vue pour les vocabulaires d\'{e}velopp\'{e}s. Cependant, une classification pr\'{e}cise des entit\'{e}s g\'{e}ographiques et topologiques est un d\'{e}but pour remplir ce principe.
 \item \textbf{P4}: \textit{L'information g\'{e}ographique n\'{e}cessaire pour la bonne gouvernance à tous les niveaux devrait \^{e}tre facilement disponible et transparente}. La publication du portail \url{http://data.ign.fr}  est l'un des objectifs afin d'avoir \'{e}galement des donn\'{e}es à la fois processable par la machine et lisible par l'homme avec l'aide des concepts et des technologies s\'{e}mantiques.
 \item \textbf{P5}: \textit{Acc\`{e}s facile pour retrouver quelle l'information g\'{e}ographique est disponible, comment elle peut \^{e}tre utilis\'{e} pour r\'{e}pondre à un besoin particulier, et dans quelles conditions elle peut \^{e}tre acquise et utilis\'{e}e.} La publication des donn\'{e}es sur le Web contribue en soit à tirer profit de leur d\'{e}couverte et int\'{e}gration . En outre, une licence explicite attach\'{e}e à aux jeux de donn\'{e}es publi\'{e}es contribue à atteindre de ce principe.
\end{itemize}

Pour les fournisseurs de donn\'{e}es g\'{e}ographiques, les avantages de publier leur donn\'{e}e sur le Web selon les principes du Linked Data sont de deux sortes :
\begin{enumerate}
 \item Tout d'abord, leurs donn\'{e}es sont interop\'{e}rables avec d'autres jeux de donn\'{e}es publi\'{e}es et peuvent \^{e}tre r\'{e}f\'{e}renc\'{e}es par des ressources externes et utilis\'{e}es comme des donn\'{e}es à r\'{e}f\'{e}rence spatiale, ce qui n'auraient pas \'{e}t\'{e} le cas si elles \'{e}taient publi\'{e}es selon les normes des infrastructures de donn\'{e}es spatiales (SDI).
 \item Deuxi\`{e}mement, l'utilisation des technologies du Web s\'{e}mantique peut aider à r\'{e}soudre les probl\`{e}mes d'interop\'{e}rabilit\'{e} qui ne sont pas encore r\'{e}solus par les normes et standards actuels dans le domaine de l'information g\'{e}ographique.
\end{enumerate}

En outre, l'agence nationale de cartographie française (IGN) dispose de diff\'{e}rents types de politiques de licence pour acc\'{e}der aux donn\'{e}es à partir de leur portail professionnel\footnote{\url{http://professionnels.ign.fr/}} (par exemple pour des fins de recherche, l'utilisation commerciale, l'acc\`{e}s à la demande, etc.), avec certains acc\`{e}s pas n\'{e}cessairement ``ouvert'' ou libre d'acc\`{e}s:  (par exemple, BD TOPO\circledR). Bien qu'il y ait une compr\'{e}hension claire des avantages de la publication et de l'alignement des donn\'{e}es sur le Web, les recherches à l'IGN sont en cours sur la mani\`{e}re de combiner les licences sur des jeux de donn\'{e}es. Deux solutions sont à l'\'{e}tude :
\begin{enumerate}
 \item Les diff\'{e}rentes politiques de licence attach\'{e}es aux jeux de donn\'{e}es: Ici, la licence attach\'{e}e est donn\'{e}e directement lors de la publication. Ainsi, par exemple, s'il s'agit d'une licence libre, le point d'acc\`{e}s SPARQL est publiquement disponible et peut \^{e}tre interrog\'{e} sans aucune restriction.
 \item L'utilisation d'un m\'{e}canisme d'acc\`{e}s aux donn\'{e}es donnant acc\`{e}s selon une liste de configuration pr\'{e}d\'{e}termin\'{e}e de graphes d\'{e}di\'{e}s, des ressources et des op\'{e}rations autoris\'{e}es. Cette solution va en droite ligne avec les propositions des chercheurs comme Rotolo et al. dans l'application de la logique d\'{e}ontique.  Cette solution sugg\`{e}re que m\^{e}me s'il y a un point d'acc\`{e}s, un module de configuration des types de requ\^{e}tes à r\'{e}aliser et des politiques d'acc\`{e}s doivent \^{e}tre d\'{e}finis pour les sous-ensembles de donn\'{e}es avec un soin particulier pour tenir compte des compositions de licences dans les r\'{e}sultats.
\end{enumerate}

Selon les principes du Linked Data, les URIs devraient rester stables, m\^{e}me si les unit\'{e}s administratives changent ou disparaissent. Cela implique l'adaptation du vocabulaire de donn\'{e}es afin de g\'{e}rer les versions des donn\'{e}es, l'\'{e}volution temporelle et la granularit\'{e} des donn\'{e}es. Cette question sera abord\'{e}e dans nos travaux futurs, comme nous travaillons sur la publication du jeu de donn\'{e}es spatio-temporelle d\'{e}crivant l'\'{e}volution des communes depuis la R\'{e}volution française. Une autre question de recherche porte sur l'automatisation de l'ensemble du processus de publication, partant des donn\'{e}es g\'{e}ographiques aux formats traditionnels (SHAPE, CSV, etc) pour arriver à des donn\'{e}es RDF pleinement interconnect\'{e}s.

La derni\`{e}re question porte sur l'utilisation de plusieurs g\'{e}om\'{e}tries pour d\'{e}crire une entit\'{e} g\'{e}ographique: des g\'{e}om\'{e}tries avec diff\'{e}rents niveaux de d\'{e}tail, avec diff\'{e}rents CRS ainsi que des choix diff\'{e}rents de repr\'{e}sentation. Cela a \'{e}t\'{e} superficiellement abord\'{e}es dans notre cas d'utilisation avec l'utilisation de deux polygones et de points pour repr\'{e}senter respectivement la surface et le centre de gravit\'{e} des communes, mais doit \^{e}tre \'{e}tudi\'{e}e en profondeur pour proposer une solution int\'{e}grant à la fois les contraintes de requ\^{e}tes et d'affichage d'informations sur des fonds de carte en fonction des besoins utilisateurs.

\subsubsection*{Visualisations g\'{e}n\'{e}riques des donn\'{e}es li\'{e}es sur le Web}
Nous pr\'{e}voyons d'utiliser un ensemble plus exhaustif des vocabulaires dans nos requ\^{e}tes g\'{e}n\'{e}riques pour d\'{e}tecter les cat\'{e}gories, en prenant la liste des vocabulaires du catalogue LOV pour alimenter l'assistant. Les propri\'{e}t\'{e}s d'agr\'{e}gation peuvent \^{e}tre \'{e}tendues afin de prendre d'autres relations s\'{e}mantiques (par exemple, prendre en compte \texttt{SKOS:exactMatch}). En outre, nous pr\'{e}voyons de faire une \'{e}valuation du prototype et le comparer à des outils connexes comme celles permettant de produire des statistiques des jeux de donn\'{e}es. Nous avons \'{e}galement besoin de quantifier quand une cat\'{e}gorie est «importante» dans un jeu de donn\'{e}es. Par exemple, est-ce suffisant pour un jeu de donn\'{e}es pour \^{e}tre class\'{e} dans la cat\'{e}gorie ``GEODATA'' avec juste dix triplets contenant des adresses? À partir de quel nombre de triplets et donc quelle proportion pourrait-on utiliser les cat\'{e}gories, donc les librairies de visualisations associ\'{e}es? Ces questions peuvent en outre \^{e}tre \'{e}tudi\'{e}es pour trouver le meilleur compromis entre le pourcentage de repr\'{e}sentativit\'{e} de certaines cat\'{e}gories et les librairies correspondantes. Un  inconv\'{e}nient de notre travail sur les visualisations est l'absence d'\'{e}valuation au niveau de l'utilisateur final, avec un protocole bien d\'{e}finit pour comprendre les besoins des utilisateurs, en se concentrant davantage sur les aspects s\'{e}mantiques que sur ceux li\'{e}s juste à l'exploration (interface web). Un travail futur naturel est d'utiliser ces \'{e}valuations et r\'{e}-adapter les applications/visualisations bas\'{e}es sur les r\'{e}sultats d'une \'{e}tude utilisateur.

\subsubsection*{Vocabulaires et LOV}
\label{sec:nextSteps}
Les travaux sur l'harmonisation des pr\'{e}fixes peut \^{e}tre \'{e}tendu dans plusieurs directions. En se limitant aux deux services que nous avons \'{e}tudi\'{e} et d\'{e}jà contribu\'{e} à harmoniser les prefixes, les prochaines \'{e}tapes possibles seraient d'automatiser autant que possible les tâches qui ont \'{e}t\'{e} faites de mani\`{e}re semi-automatique à ce jour :
\begin{itemize}
 \item \emph{i)} le d\'{e}veloppement d'une interface unique pour soumettre les espaces de noms et les pr\'{e}fixes aux deux services;
 \item \emph{ii)} la couverture simulatan\'{e}e des prefixes dans LOV et prefix.cc pour harmoniser les URIs des vocabulaires pr\'{e}sents dans les deux services afin de ne confondre les utilisateurs et les d\'{e}veloppeurs dans le choix ou la gestion des espaces de noms, et ainsi proposer une liste recommand\'{e}e namespaces-URIs des vocabulaires les plus importants.
\end{itemize}

Ce dernier aspect va au-delà du domaine d'application des deux services car une telle liste pourrait \^{e}tre consolid\'{e} et approuv\'{e} par les principaux acteurs de publication et de gestion du vocabulaire, et recommand\'{e} pour une utilisation dans les applications de donn\'{e}es li\'{e}es. Cela pourrait \^{e}tre pris en charge par le prochain groupe de travail du W3C\footnote{\url{http://www.w3.org/2013/05/odbp-charter.html}} charg\'{e} de g\'{e}rer les vocabulaires dans le cadre de la nouvelle activit\'{e} de gestion des donn\'{e}es sur le Web.

Pour le ranking des vocabulaires, nous souhaitons dprendre en compte les axiomes d'\'{e}quivalence (entre les classes et les propri\'{e}t\'{e}s) lors du calcul du contenu de l'information, et plus g\'{e}n\'{e}ralement, toutes sortes de relations s\'{e}mantiques entre les termes. En outre, nous pr\'{e}voyons de comparer notre mod\`{e}le de classement avec d'autres approches telles que celles de classement fond\'{e}es sur les graphes  (par exemple, le PageRank). Une autre orientation future est de chercher la d\'{e}pendance dans la position entre les vocabulaires, en se concentrant sur un type sp\'{e}cifique de ``liens entrants'' (à savoir les extensions, les g\'{e}n\'{e}ralisations) et d'\'{e}tudier comment ils affectent les m\'{e}triques que nous avons pr\'{e}sent\'{e} dans cette th\`{e}se.

Nous avons fait l'hypoth\`{e}se dans cette th\`{e}se que l'acc\`{e}s aux donn\'{e}es se faisait soit en interrogeant un acc\`{e}s SPARQL, en parcourant le graphe par le principe du ``follow-your-nose'' ou en t\'{e}l\'{e}chargeant des dumps. R\'{e}cemment, une nouvelle façon d'acc\'{e}der aux donn\'{e}es sur le Web est en train d'\'{e}merger: à travers  des motifs de fragments triplets li\'{e}s\footnote{\url{http://linkeddatafragments.org/}}. Ce concept vise à explorer les acc\`{e}s de donn\'{e}es avec des fragments simples de donn\'{e}es pour r\'{e}soudre les requ\^{e}tes côt\'{e} client avec les donn\'{e}es h\'{e}berg\'{e}es dans un serveur. Les serveurs peuvent servir des donn\'{e}es à faible coût de traitement d'une mani\`{e}re qui permet l'interrogation côt\'{e} client en d\'{e}plaçant du m\^{e}me coup l'intelligence passant du serveur vers le client. Un travail futur pourrait \^{e}tre d'utiliser le concept côt\'{e} client pour \'{e}valuer les points d'acc\`{e}s aux donn\'{e}es contenant seulement les g\'{e}om\'{e}tries structur\'{e}es ou litt\'{e}raux pour les applications du monde r\'{e}el. Enfin, le concept de fragments de triplets  peut \'{e}galement \^{e}tre appliqu\'{e} pour d\'{e}tecter des pattern pour la visualisation des diff\'{e}rents point d'acc\`{e}s de donn\'{e}es.

Avec la croissance continue et soutenue de la publication des donn\'{e}es ouvertes et li\'{e}es sur le Web, il en sera aussi des jeux de donn\'{e}es et des ontologies sur des donn\'{e}es g\'{e}ospatiales. Les producteurs de donn\'{e}es  g\'{e}ographiques vont continuer de lib\'{e}rer de plus en plus fr\'{e}quemment leurs donn\'{e}es sur le Web. Cela va cr\'{e}er un besoin d'outils pour facilement cr\'{e}er des analyses, en particulier dans l'extraction et la fouille de gros volumes de donn\'{e}es pour retro-alimenter les \'{e}diteurs quant à l'utilisation effective des triplets. La gestion des flux de donn\'{e}es g\'{e}ospatiales sur le Web va demander des impl\'{e}mentations plus efficaces dans le domaine des donn\'{e}es spatiales pour \^{e}tre en mesure d'interroger à la vol\'{e}e des donn\'{e}es contenant aussi de l'information temporelle. Ainsi, la mod\'{e}lisation des flux de donn\'{e}es en streaming g\'{e}o-temporel, l'interrogation et l'analyse sur le Web sont susceptibles d'\^{e}tre les prochains d\'{e}fis que les technologies du Web s\'{e}mantique devront faire face et r\'{e}soudre.

\end{document} 