\chapter{Survey on Visualizations for RDF}
\label{ch:ch4}

\begin{flushright}
\textit{``A Semantic Web application is one whose schema is expected to change.''} \\ (David Karger, MIT)\footnote{A statement during his keynote at ESWC 2013 conference}

\end{flushright}

\textcolor{red}{Content: 
 Tools for visualizations: facete, soA, etc.
 Survey on visualizations applications
 Explore limitations
 Find answers
\end{itemize}}

\section{Tools for visualizing RDF Data}
\label{sec:related}
%\ghis{add comparison among different SoA work - add Tabulator ref}\\

There are currently many projects aiming at visualizing (RDF) Linked Data. A survey by Dadzie and Rowe \cite{Dadzie:2011} concluded with the fact that many visualization tools are not easy to use by lay users. In~\cite{Klimek2014}, there is a recent review of some visualizations tools that can be summarized as follows:
\begin{itemize}
 \item \textit{Vocabulary based visualization tools:} these tools are built for specific vocabularies and that help in visualizing data modelled according to those vocabularies, such as CubeViz \cite{cubeviz:2012}, FOAF explorer\footnote{\url{http://foaf-visualizer.gnu.org.ua/}} and Map4rdf \cite{leon2012}. They aim at visualizing data modelled respectively with \texttt{dq,foaf} and \texttt{geo+scovo}.
 \item \textit{Mashup tools:} they are used to create mashup visualizations with different widgets and some data analysis, such as DERI Pipes \cite{danh2009}. Mashup tools can be integrated into the LD wizard to combine different visual views.  
 \item \textit{Generic RDF visualization tools:} they typically support data browsing and entity rendering. They can also be used to build applications. In this category, we can mention Graphity\footnote{\url{https://github.com/Graphity/graphity-browser}}, lodlive\footnote{\url{http://en.lodlive.it/}} and Balloon Synopsis\footnote{\url{https://github.com/schlegel/balloon-synopsis}}.
\end{itemize}
While these tools are often extensible and support specific domain datasets, they suffer from the following drawbacks:
\begin{itemize}
 \item \textit{They are not easy to set up and use by lay users}. Sometimes, users just need to have a visual summary of a dataset in order to start exploring the data. Our approach to this challenge is to provide such a lightweight javascript-based tool that supports a quick exploration task.
 \item \textit{They do not make recommendation based on categories}. A tool similar to our approach is Facete\footnote{\url{http://cstadler.aksw.org/facete/}}\cite{facete:2014} which shows a tree-based structure of a dataset based on some properties of an endpoint more target at geodata. A tabular view enables to visualize slices of data and a map view can be activated when there is geo data. Our approach aims to be more generic, offering more views (tabular, map, graph, charts, etc.) according to a systematic analysis of what are the high level categories present in a dataset.
\end{itemize}

Regarding wizard-based tools for visualizing data, similar approaches are available for tools consuming datasets in CSV/TSV. ManyEyes\footnote{\url{http://www-958.ibm.com/software/data/cognos/manyeyes/}} is an IBM online tool that suggest charts according to the columns of a given CSV file. Similarly, Google Charts\footnote{\url{https://developers.google.com/chart/}} help to achieve the same goals for creating embeddable charts by using DSP language in the framework. Datawrapper\footnote{\url{http://datawrapper.de/}} is an open source tool to enable the creation of basic charts originally target at journalists inspired by ManyEyes and Google Charts. All the visualizations are based on the type of the columns/fields of the data. In Linked Data, vocabularies are used for modeling datasets in RDF, thus making it difficult to reuse directly those tools. The benefit of our approach is that it constructs specific SPARQL queries to detect the presence or not of predefined specific types of information, yielding to information type-specific visualisations to enable end users to quickly start to explore dataset in a generic manner. 
%similarly in the generation of charts, by using directly existing libraries consuming RDF datasets and standard 

\section{Visualization Applications Over Linked Data}
\label{sec:visAppsLD}

As many initiatives on Linked Open Data is growing, tools and technologies are getting more and more mature to help pro-consumers to leverage the lifting process of the data. At the same times, standardization bodies such as W3C are helping in providing best practices to publish Open Government Data by using appropriate vocabularies, taking care of stability in the URIs policies, and making links to other datasets. It is the case for example of the Government Linked Data Working Group\footnote{\url{http://www.w3.org/2011/gld/}} which is developing standards to help governments publishing their data using Semantic Web technologies. Having a look at different proposals of the Life Cycle of Government LD, one of the last stage is ``Publication'' where the data is released according to the 4-5 stars principles\footnote{\url{http://5stardata.info/}}, with a given access to a SPARQL endpoint. However, for a better understanding of the data, one of the next step is usually to building visualizations through intuitive charts, graphs , etc. that will benefit to citizens, data journalists and other public authorities to improve the quality of their decisions. At the moment, one way of doing is to look around previous initiatives to see what type of application is already there, and make something similar according to a given dataset. Another approach is by organizing \textit{contests} where the challenges are to mash up unexpected datasets with clear and beautiful visualizations. Such approach is harder as developers also try figure out which tool and library is used for different applications. What if we describe applications according to the facets/views, datasets, visual tools etc.? 
%In this paper, we propose a small vocabulary aiming at describing applications developed on top of LOD for more interoperability and components reusability. 
%This paper is organized as follow: Section \ref{sec:apps} defines the types of applications on LOD, followed by a framework for assessing tools for building such applications (cf. Section \ref{sec:frame}). Section \ref{sec:reusable} deals with the facts which are common to many visual applications, and Section \ref{sec:dvia} presents the vocabulary for describing the semantics of mashups. We make some references to similar works in Section \ref{sec:rworks} and provide some outlook in Section 

\section{Catalog of Applications on the Web} \label{sec:rworks}
% Mention the efforts of UNiv Southampton for describing the apps developed cfr http://data.southampton.ac.uk/dumps/apps/2012-12-18/apps.ttl
% Mention the efforts at RPI for some of the apps developed for data.gov.
\textcolor{red}{re-read paper approaches to visualising Linked Data: A survey (Aba-Sah and M. Rowe)} \\

The Open Data Service at the University of Southampton\footnote{\url{http://data.southampton.ac.uk/apps.htm}} has a register of all the applications developped using their datasets. A catalog of the Applications using the data is available at \url{http://id.southampton.ac.uk/dataset/apps}. Each application is described by giving the name, the type of the application, date of creation, the creator and the datasets used. There is also a flag to specify if the application is ``official'' or not. This initiative seems to be isolated and by having a common layer of vocabulary for the applications, we could add more informations as it is intended in DVIA vocabulary. 

Another approach at Rensseler Institute\footnote{\url{http://data-gov.tw.rpi.edu}} is to put at the bottom of the static page of the demo/application showcasing the benefits of Open Data for data.gov. is to put information, containing also a link to the SPAQRL query used for generating the application. As this information is human-readable and can help, the main drawback is the lack of a machine-readable version, using semantics. DVIA can leverage the issue aggregating many such descriptions in other Open Data initiatives as well. 

Regarding the tools for visualizing Linked Data,  the paper \cite{aba2011} analyses in detail the current approaches used to browse and visualize Linked Data, by identifying requirements for users classify into two groups: tech-savvy and lay-users. As the authors extensively surveyed more generic Linked Data browsers, with text-based presentation and visualization options, they provide some recommendations according to the size of the data such as fine-grained analysis among others. However, they do not target their study on tools that can easily help building visual SW-based applications. However, our approach is to study the tools used to build innovative applications for detecting the components that could be reusable across different domain y/o scope. 

\section{Linked Data Applications} \label{sec:apps}
According to \cite{card99}, \texttt{Visualization} is \textit{ ``the use of computer-supported, interactive visual representations to amplify cognition''}. So the unique object of visualization is developing insights from collected data. That justify why each time a new dataset is released, users always expected some showcases to play with the underlying datasets. It is true that many public open initiatives uses incentives actions like \textit{challenges}, \textit{datahackday} or \textit{contest}, etc. to find innovative applications that actually exhibit the benefits of datasets published. Visualizations play crucial role as they can easily find errors in a large collection; detect patterns in a dataset or help navigate through the dataset. 

\subsection{Typology of Applications}
Jeni Tennison\footnote{\url{http://www.theodi.org/people/jeni}} defines  in her blog\footnote{\url{http://www.jenitennison.com/blog/node/126}} three categories of applications using online data:
\begin{itemize}
\item (i) \textit{data-specific applications}, which are constructed around particular data sets that are known to the developer of the application; hence the visualizations obtained are of data-specific applications. Examples are the famous applications of ``Where does my money go''  in Greece\footnote{\url{http://publicspending.medialab.ntua.gr/en/index.php}} or UK\footnote{\url{http://wheredoesmymoneygo.org/}}. Those applications are also called ``mashups". 
\item (ii) \textit{vocabulary-specific applications}, which are constructed around particular vocabularies, wherever the data might be found that uses them. Examples here are FaceBook� Social Graph API , IsaViz\footnote{\url{http://www.w3.org/2001/11/IsaViz}}, among others
\item (iii) \textit{generic applications}, which are constructed to navigate through any RDF that they find; e.g.Tabulator\cite{tabulator06}, OpenLink Data Explorer\footnote{\url{http://ode.openlinksw.com/}}

\end{itemize}
Because most mashups are data-specific applications, it is important and necessary to  know what information the dataset contains. This could be achieve by giving the meaning of some properties or classes of the vocabularies used to create the dataset. Hence, what the data publisher needs to do very often is to make sure that the data they publish is documented. However, what is seeing in practice, is to consider using an intuitive visualization self-descriptive to both show the added-value of the data and its documentation.

\subsection{Libraries } \label{sec:frame}
We have surveyed a number of visualization tools - fifteen to be precise- used to build applications over data in general (raw or structured) based on the following features:
\begin{itemize}
\item \textit{Data Formats} for the format of data taken as input by the tool;
\item \textit{Data Access}, for the way to access the data from the tool, such as web service, sparql endpoint, etc.
\item \textit{Language code}, the programming language used to develop the tool;
\item \textit{Type of Views}, the different views potentially accessible when using the tool;
\item \textit{Imported Libraries}, the external libraries available within the tool,
\item \textit{License} for the Intellectual Properties rights of the tool,
\item \textit{SemWeb Compliant}, whether the tool can be easily transposed or compliant with structured data; and 
\item \textit{Creator}, organizations or persons who developed the tool.

\end{itemize}
\todo{ADD Visual Box in the survey of Table \ref{tab:visuTools} }

The outcome of this state-of-the-art can then be used to assess in the choice of a given visualization tool, according to some criteria, such as (i) usability, (ii) visualization capabilities, (iii) data accessibility, (iv) deployment and (v) extensibility. For more details on this survey, the readers are encouraged to read \cite{deliverable2012b}.\footnote{Note: The survey does not contain Visual Box as it was not released at the time of writing the survey}. Table \ref{tab:visuTools} gives an overview of the selected tools studied.

\begin{landscape}
\begin{table}[ht]
\centering
\begin{tabularx}{\linewidth}{ |X|X|X|X|X|X|X|X|X|}
\hline
\textbf{Tools} & \textbf{Data Formats}& \textbf{Data Access} & \textbf{Language Code} & \textbf{Type of Views} & \textbf{Imported Libraries} & \textbf{License} & \textbf{SemWeb Compliant} & \textbf{Creator} \\
\hline
Choosel & xls, csv & API & GWT & Text, Map, Bar chart & Time (Simile), Protovis, charts, Flexvis & Open & No & Lars Grammel\\
\hline
Fresnel & RDF & --& RDF& Propertie,  Labels & Welkin, IsaViz, Haystack, CSS &  Open & Yes & Emmanuel Pietriga et al. \\
\hline
Spark & RDF-JSON & SPARQL& PHP& Date chart, Pie chart, simple table &  -- &  Open & Yes & AIFB-KIT \\
\hline
LDA & RDF & SPARQL& Java, PHP& -- &  -- &  Open & Yes & Talis, Epimorphis \\
\hline
Semantic Web Import & RDF & SPARQL CONSTRUCT& Netbeans& Graph Node Views &  -- &  CECILL-B & Yes & Wimmics (INRIA) \\
\hline
Many Eyes & xls, plain text and HTML & API& Java, Flash& charts, trees, graphs, maps & --&  IBM & No & IBM research \\
\hline
D3.js & CSV, SVG, GeoJson & API& JavaScript& charts, trees, graphs, maps & Jquery, sizzle, colorbrewer& Open & Possible & Mike Bostock \\
\hline
Facet Spatial Semantic Browsing Widgets & RDF- JSON & SPARQL& JavaScript& Map,Facet view & Jquery, dynatree&  Open & Yes & AKSW \\
\hline
Sgvizler & RDF- JSON & SPARQL SELECT& JavaScript& Map, Line chart, timeline, sparkline & Google visualization API&  Open & Yes & Martin G. Skjaeland \\
\hline
Visual Box & RDF & SPARQL SELECT& PHP, Django template& Map, charts, timeline, graphs & Google charts, TimeKnots, d3.js&  Open & Yes & Alvaro Graves (RPI) \\
\hline
Map4rdf & RDF- JSON & SPARQL & Java, GWT& Facet, Map & OSM Layers, Google Maps&  Open & Yes & OEG (UPM) \\
\hline
Exhibit & JSON Exhibit & Data dump& JavaScript& Map,Tile, Thumbnail, Tabular and Timeline & ---&  Open & Yes & MIT \\
\hline
Google Visualization API & JSON, CSV & API& JavaScript& Many charts, controls and dashboard & AJAX API&  Open & Possible & Google \\
\hline
Data Publica & DSPL & API& JavaScript& Map,graph, histogram, table & Google charts, highchart.js& Proprietary & No & Data Publica \\
\hline
GeoAPI & GML, KML, GPX & API& JavaScript& Map views & OpenLayers, Prototype.js& Free for non commercial use & Yes & IGN (France) \\
\hline
\end{tabularx}
\caption{Survey of some tools used for making mashups on the Web.}
\label{tab:visuTools}
\end{table}
\end{landscape}

\subsection{On Reusable Applications} \label{sec:reusable}
%\todo{ summary of the application survey D6.1: diversity, scope, and countries} 
%\todo{show with the sample of openspending.fr} \\
%\todo{ The choice of the features used}
We first review the numerous applications that have been developed on top of datasets  opened by governments (UK, USA, France) and public local authorities. We made a random survey of thirteen (13) applications \cite{deliverable2012a} in various domains such as of security, health, finance, transportation, housing, city, foreign aid and education. The main template used was to find out:
\begin{itemize}
\item the name of the application:
\item the scope or target domain of the application;
\item a small and concise description;
\item the platform on which the application can be deployed and view;
\item the policy used for creating the URL of the application;
\item the legacy data used to build the application, and a mention of the process of the lifting process of the raw data to RDF if available;
\item the different views available of the application;
\item comments or relevant drawback to mention;
\item the license of the application
\end{itemize}
 Table \ref{tab:describeApps} provides the information extracted from \texttt{openspending in Greece} using the aforementioned template. 


\begin{table}[ht!bp]
    \caption{Gathering reusable information from openspending in Greece Application} \label{tab:describeApps}
    \small
    \center
    \begin{tabularx}{\textwidth}{@{}lX@{}}
     %\begin{tabular}{@{}llX@{}}
    \toprule
    \textbf{Features} & \textbf{Value}\\
    \toprule
    \texttt{Access Url }&	\url{http://publicspending.medialab.ntua.gr/}\\
    \midrule
    \texttt{Scope/Domain} &	Public spending, Government \\
    \midrule
    \texttt{Description} & The application helps visualizing the most characteristic facts of the Greek public spending, interconnected to foreign expenditure and other data. \\
    \midrule
    \texttt{Supported Platform} &	Web \\ 
    \midrule
    \texttt{URL Policy}   &  http://{BASE}/en/{NAME-CHART}.php e.g., \url{http://{BASE}/en/toppayersday.php} \\
    \midrule
    \texttt{Data Source}	& \url{http://opendata.diavgeia.gov.fr}; Greek Tax data (TAXIS) \\ 
    \midrule
\texttt{Type of views} & Bubble tree, column and bar charts \\ 
    \midrule
   \texttt{Visualization tools} &  HighchartsJS,  Bubble TreeJS JqueryJS ; RaphaelJS \\ 
   \midrule
  \texttt{License} & Open \\ 
    \midrule
\texttt{Business Value} & Not Commercial (Free) \\ 
    \bottomrule
  
    % \end{tabular}
    \end{tabularx}
    \end{table}
   
 \section{Evaluating Visualizations}
 See Roberto Garcia work and proposal in Datalift technical report.

  \section{Summary}
  
